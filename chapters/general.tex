
\chapter{研究生学位论文撰写的总体要求}
密集的多用户环境中的无线电发射器识别是优化频谱利用,识别和最小化干扰以及执行频谱策略的重要工具。 无线电数据是容易获得的并且易于从天线获得,但是标记和策划的数据通常很少,使得监督式学习策略在实践中困难且耗时。 我们证明,半监督学习技术可以用来扩大学习范围以外的监督数据集,允许识别和召回新的无线电信号通过使用基于非监督和监督方法的非线性特征学习和聚类方法的稀疏信号表示。\par

在密集和复杂的多用户频谱环境中的无线电信号识别是优化频谱利用,识别和最小化干扰,执行频谱策略以及实施有效的无线电感知和协调系统的重要工具。该问题的经典方法侧重于能量检测以及使用专家特征和判定标准来识别和分类特定的调制类型[2] [1]。这些方法依赖于信号属性,特征和决策统计的先验知识来分离已知的调制,并且通常在简化的分析硬件,传播,无线电环境模型下导出。
我们最近证明了有监督的无线电分类系统的天真特征学习的可行性[14],考虑到标记数据集和例子,允许联合特征和分类器学习。在这种情况下,我们能够超越传统的基于专家决策统计的敏感性和准确性分类的显着差距。这是一个强有力的结果,对当前的解决方案提供了显着的性能改进,但它仍然完全依赖于监督学习和精心策划的培训数据。在现实世界中,特别是在无线电领域,我们面临着大量的未标记的示例数据可用于我们的传感器和不完整的地面真相的类标签知识。为了解决这个问题,我们研究了无线电识别学习的替代策略,这些策略对标记的训练数据的依赖程度较低,并且能够使用无标记或较少标记的例子来理解无线电信号,这可能极大地减轻了在这种机器学习系统供开发人员和维护人员使用,并允许系统识别新的信号并扩展以了解新环境。\par

在半监督学习[9] [8] [5]中,我们试图分离和识别新的类,没有明确的类标签,这些类的例子允许我们专门学习特征来分离它们。为了解决这个问题,我们考虑将信号示例的维数降低到一个平滑的较小的空间,我们可以执行信号聚类。给定一个适当的维数降低,我们寻求一个空间,相同或相似类型的信号具有较低的距离,而不同类型的信号间隔较大的距离。理想情况下,在这样的空间中,相同或相似类型的例子形成离散且可分离的簇,彼此容易辨别。在这样的空间中信号类型的分类则成为识别聚类,将标签与每个聚类相关联(而不是每个例子,一个劳动密集度较低的任务)的问题,以及允许识别,分类和历史注释的问题新的班级和新的班级例子,即使没有标签的知识。\par

我们专注于使用两种形式的卷积神经网络来学习原始采样无线电信号时间序列示例的稀疏表示。我们利用RadioML16.04 [15]标记的无线电调制数据集进行11种调制,其中包括白噪声,振荡器漂移,采样时钟漂移和衰落的影响。这个数据集的例子如图1所示。A.纯粹无监督的稀疏表示在没有使用类标签的情况下,我们采用纯粹无监督的方法来学习数据集的稀疏表示。这可以通过应用基于依赖关系的降维技术(例如主成分分析(PCA)或独立分量分析(ICA))来完成,然而相反,我们利用输入信号的重构通过稀疏表示在自编码器神经网络中学习的一组卷积基函数[3]。自动编码器是无监督的学习构造,其中神经网络的优化目标是通过一些更有约束的维度的中间表示来最小化输出处的重构误差以匹配输入,通常使用均方误差(MSE)损失函数并且将随机梯度下降的形式用作求解器,通过从该损失项向后传播梯度以找到接近等式III-A中的最佳网络参数。我们使用RMSProp [11]和Adam [12]梯度下降求解器进行优化,两者都获得了相似的结果。\par
通过约束网络中的中间层宽度,从而可以以低误差重建原始全维例子,通过提取用于聚类的中间稀疏编码来获得非线性降维。在这种情况下,使用类似的基函数和脉冲形状的调制可以由类似形状的卷积滤波器和编码器和解码器内的中间特征映射来表示,使得它们存在于该压缩空间的相似区域中。自编码器中的卷积层由于它们在时间上的不变表示性质以及受约束的参数搜索空间相对于它们的完全连接层等价物而非常适合于无线电时间序列信号表示。我们使用丢失[13]和输入噪声[7]作为正则化技术来帮助改善我们在数据集上学习表示的泛化。图2显示了我们的卷积自动编码器使用的体系结构。\par
在自编码器的训练过程中,我们尽量减少重构均方误差(MSE),但由于我们的主要目标是获得一个良好的聚类稀疏表示,我们显着限制隐藏层维度到一个点,我们的重构作出一些可见的简化假设,在最优重构误差下降低了隐层维数。图3显示了两个训练样例,2x128输入向量的样子,1x30稀疏表示的样子,以及2x128输出重构的样子。这给了这个网络的学习代表能力的一些直觉。\par