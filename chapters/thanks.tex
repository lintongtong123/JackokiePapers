
\begin{thanks}

时光荏苒,如月如梭。\par

转眼之间,研究生三年已接近尾声,回想起过去的这三年,有成长当然也有遗憾,
但在此还是要感谢研究生陪伴我三年的老师、同学以及朋友们。\par

首先,衷心地感谢我的导师杨清海教授。
在读研期间,他给我提供了很多好的科研条件。
从研一开始,他便每周陪我们一起做一次周报,知道我们如何读论文,如何去做PPT,如何去把自己学到的东西给大家讲出来,
这对于我的研究生生涯是一个很大的帮助,让我懂得了如何去学习并将自己的成果展现出来。
生活上,杨老师也给与了我们很多无微不至的关怀。他经常会询问我们的近况,给与我们一定的帮助与指导;
每年的春天,也会让我们走出实验室,体会大自然的风光,增强了我们团队的凝聚力。
是他教会了我,做人善事然后科研,让我懂得了一些基本原则,这对于我以后的帮助也是非常大的。
真心挺感谢杨老师这三年的谆谆教诲,临别之际,他虽远在大洋彼岸,也望风儿能传信至杨老师心里。\par

我还要感谢沈中、刘明骞和王勇超老师。在我的科研项目中,沈老师给与了很多支持,并给与正确的指导,才让我有了一定的进步。
在我的毕设过程中,刘老师对算法逻辑合理性、数据合理性给与了很大的帮助,让我能够顺利完成我的毕业设计。
数据的获取是我毕设的关键,王老师在整个过程中,从理论上给与了我很大的帮助,并指导我如何去使用我们的设备,
对于我的毕设的数据基础提供了非常大的帮助。
在此,我对帮助我的各位老师们致以诚挚的敬意并表示衷心的感谢。

还要感谢在我论文成稿中给我提供了许多帮助的秦剑、张劭、李万、唐靖旋、刘祥等人。
在项目中我们互帮互助,在论文的写作仿真中我也受到启发,这很大程度上促进了我的进展并提高了效率。 \par

特别感谢我的父母,虽然远隔千里,虽然我早已成人,但他们仍是我精神上的支柱,是我在遇到困难时继续前进的动力。\par

最后,衷心地感谢为评阅本论文而付出辛勤劳动的专家和教授们! \par

赵纪伟\par

2018 年 4 月于西安电子科技大学\par

\end{thanks}
