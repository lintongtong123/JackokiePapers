\chapter{基于CAE-CNN的无线信号调制识别}

\section{引言}

无线通信领域的研究人员已经开始将深度神经网络应用于认知无线电,并取得了一定的成果[13] [12][10]。[Tim Oshea]最近证明了利用原始数据进行有监督无线调制识别[14]的可行性,作者利用原始信号经希尔伯特变换后得到的$I$与$Q$路信号作为训练样本,调制方式作为标签,训练CNN分类器。结果显示,其分类性能超越了传统的基于专家特征的决策树、SVM等分类模型。然而,作者仅仅是用了传统的CNN框架,并没有对分类性能以及网络框架进行进一步的研究。\par

本章针对无线信号调制识别问题,提出了一种基于卷积自编码器(CAE)与卷积神经网络(CNN)融合的无线信号调制识别的算法框架,并将此框架下的识别准确率及鲁棒性等与传统的基于特征的识别方法进行比较分析。\par


\section{调制信号生成}

无线接收端的信号实际上是信号经过信道作用得到的。尽管在机器学习中我们一般会建议使用真实数据,但是在无线电通信领域中,由于标记数据匮乏,而且受到多径等效应的影响,很难直接使用真实数据进行训练。我们利用XXX仿真仪,构建信号生成框架,通过多径信道和高斯信道来获取近似真实的仿真信号。\par


\subsection{信道建模}

无线信道模型是对无线信道的抽象描述,它能很好地反映真实环境中的信号传输规律。无线通信数据信息主要通过无线信道作为载体通过无线信道传输。 由于无线信道的环境复杂多变,电波以不同的传输方式(直射,反射,散射等)到达接收点,使得接收到的信号与发射的信号不同。因此,只有准确预测无线信号的无线传播特性,如路径损耗和相位延迟,才能为无线网络提供合理的设计,部署和管理策略。\par

无线信号的调制识别可以看作是一个N类的决策问题。其中,我们的输入是一个接收信号的复时间序列。
也就是说,我们以离散时间步长对无线电信号的同相和正交分量进行采样,以获得1×N的复数值向量。\par

\begin{equation}\label{sec:eqt3_1}
r(t) = s(t)*c + n(t)
\end{equation}

我们将接收信号用等式\ref{sec:eqt3_1}表示。其中,将连续信号或一系列离散的时间序列信号,调制到具有变化的频率、相位、振幅、或多个变换的正弦波上,得到调制信号$s(t)$。
 $c$是信号上的一些路径损耗或恒定增益项,$n(t)$是反映热噪声的加性高斯白噪声过程。从解析的角度来看,这个简化的表达式在专家特征和决策统计的发展中被广泛使用。\par

然而,实际的信道环境却比较复杂。发射信号$s(t)$,在传播过程中经历多个信道效应, 最后在接收端被接收为$r(t)$。这些信道效应包括:时间延迟,时间缩放,相位旋转,频率偏移,加性热噪声以及与信号卷积的信道脉冲响应,以及所有的随机时变过程等。 这些效应对信号的作用可以近似表示为:\par

\begin{equation}\label{sec:eqt3_2}
	r(t) = e^{j*n_{Lo}(t)} \int_{\tau=0}^{\tau_{0}} s(n_{Clk}(t-\tau))h(\tau) + n_{Add}(t)
\end{equation}

这考虑了许多对于模型来说很重要的现实世界的影响:通过残留载波随机游走过程调制$n_{Lo}(t)$,通过残留时钟振荡器随机游走重采样$n_{Clk}$,与时变的旋转非恒定幅度脉冲响应$h(t-∞)$卷积,以及加性噪声$n_{Add}(t)$(可能不是白噪声)。 每个都可能导致未知的时变误差。考虑到现实世界中存在的无线信道的影响时,这使传输数据表示与原始数据表示复杂化。\par

考虑到传播信道的复杂性,对专家特征提取并进行分类决策建模是很难的。这通常会迫使我们简化假设,构建易于处理的如方程\ref{sec:eqt3_1} 所描述的基本模型;然而,基本模型很难刻画复杂的信道特征,这样就造成了算法性能的上限较低,鲁棒性较差。在本文中,我们主要关注包括所有上述影响的模拟传播环境中的实测数据,利用数据反映信道本身的特征,而不是从理论上的进行信道建模指纹特征提取等。\par

\subsection{调制信号获取}
% 为了简化信号的生成,我们使用XXX信号生成器自带的信道模型,这包括许多所需的信道,例如高斯信道,频率选择性信道、多径信道、瑞利信道等。下文当中使用的数据,我们是利用高斯信道生成的数据。\par

无线电通信信号实际上是综合生成的,而且我们以与真实系统完全相同的方式确定性地引入调制,脉冲整形,携带数据以及与现实世界信号相同的其他充分表征的发射参数。 我们将真实的语音和文本数据集调制到信号上。 在数字调制的情况下,我们使用块随机数发生器对数据进行白化以确保比特等概率。\par

无线电频道效应的特征相对较好。 我们采用鲁棒模型来处理信道脉冲响应的时变多径衰落,载波频率振荡器和采样时钟的随机游走漂移以及加性高斯白噪声。 我们通过我们的合成信号集合通过严酷的信道模型,将未知的尺度,平移,膨胀和脉冲噪声引入我们的信号。\par

我们使用GNU无线电信道模型[14]模块在GNU Radio [3]中对这个数据集的生成进行建模,然后使用128个样本矩形窗口过程将每个时间序列信号切片成测试和训练集。 总数据集大约500 MBytes作为一个python pickle文件存储与复杂的32位浮点样本。\par

这些数据有望成为该领域其他人的极大用处,并可以作为这一领域的基准。 这个数据集可以在http:// radiom.com上以pickle python格式获得,由时间窗口的例子和相应的调制类别和SNR标签组成。 我们希望扩大调制范围和渠道现实主义作为这方面的兴趣。\par

我们专注于由11个调制组成的数据集:8位数字调制和3位模拟调制,都被广泛应用于我们周围的无线通信系统。 这些包括BPSK,QPSK,8PSK,16QAM,64QAM,BFSK,CPFSK和PAM4或数字调制,以及用于模拟调制的WB-FM,AM-SSB和AM-DSB。 数据以大约每个符号8个采样的速率进行调制,标准化的平均发送功率为0dB。\par

为了生成一个表征良好的数据集,我们选择一系列在实践中广泛使用的调制,并以离散二进制字母(数字调制)和连续字母(模拟调制)进行操作。我们调制每个调制解调器上的已知数据,并使用GNURadio将它们分别暴露于上述频道效应。我们将数百万个样本分割成由许多短时窗组成的数据集,其方式类似于连续的声学语音信号通常为语音识别任务而开窗。我们提取了64个样本的128个样本的步骤,以形成我们提取的数据集。\par

在分割之后,假设采样率大约为1MSamp/sec,则示例大致为128μs。每个包含8到16个带有随机时间偏移,缩放,旋转,相位,通道响应和噪声的符号。这些例子表示关于调制数据比特的信息,关于它们如何被调制的信息,关于在传播期间信号通过的信息,以及关于发送和接收器设备状态以及包含的随机过程的状态的信息。我们专注于恢复关于信号如何被调制的信息,从而根据与调制方案相对应的离散的11类标签集来标记数据集。\par


自然图像有其固有特性,也就是说,图像的一部分的统计特性与其他部分是一样的。这也意味着我们在这一部分学习的特征也能用在另一部分上,所以对于这个图像上的所有位置,我们都能使用同样的学习特征。\par

我们专注于使用两种形式的卷积神经网络来学习原始采样无线电信号时间序列示例的稀疏表示。我们利用RadioML16.04 [15]标记的无线电调制数据集进行11种调制,其中包括白噪声,振荡器漂移,采样时钟漂移和衰落的影响。这个数据集的例子如图1所示。\par


\section{调制信号的表示}

不同的调制信号具有不同的时频特征。本节,我们将原始数据可视化,来了解不同信号的时频特征;同时,我们利用CAE以及CNN获取信号的无监督表示,并展示不同信号在CAE特征空间中的分布状况,从而进一步了解不同网络框架对调制信号进行特征提取的不同状况。

\subsection{数据集可视化}

对于每一种调制方式,我们随机抽出一个样本,并对其时域(图1)和频域(图2)进行研究。我们可以发现,不同调制方式之间具备许多相似性和差异性。但是,由于脉冲形变,失真和其他信道影响,人类专家也很难从视觉上分辨各个调制类别。\par

\begin{figure}[!h]
	\centering
	\includegraphics[scale=0.9]{figures/chapter_3/signal_view_1}
	\caption{不同调制方式的高SNR样本的时域波形}\label{fig_3_2}
\end{figure}

在时域中,我们可以看到XX信号具备较明显的特征,而XXX特征在视觉上让人感觉像是噪声,很难直接判断出来。\par

\begin{figure}[!h]
	\centering
	\includegraphics[scale=0.9]{figures/chapter_3/signal_view_2}
	\caption{不同调制方式的高SNR样本的功率谱}\label{fig_3_3}
\end{figure}

在频域中,每一个信号都具备一个带宽限制的功率包络,其形状为调制识别提供了一定的信息,但是对于人类专家来说,这却是一个很困难的繁琐的视觉判定方法。\par

\subsection{调制信号的无监督表示}

无监督的稀疏表示是指在没有使用类标签的情况下,利用无监督的方法来学习数据集的稀疏表示。这可以通过使用基于数据依赖性的降维技术来完成,例如主成分分析(PCA)或独立分量分析(ICA)。但是,这些方法只能对数据进行线性降维。\par

卷积自动编码器非常适合于减小参数空间,获取的卷及特征具有时移不变性。 本文中,我们利用卷积自编码器对输入的信号进行重构,学到一组原始信号的非线性稀疏表示。\par

自动编码器是一种无监督的学习算法,其中神经网络的优化目标是通过一些更有约束的中间维度,使用均方误差(MSE)等损失函数,最小化输出处的重构误差。通常,自编码器利用反向传播算法,将误差进行反向传播,并使用随机梯度下降(SGD)算法等,以找到接近等式\ref{sec:eqt3_3}中的最佳网络参数。

\begin{equation}\label{sec:eqt3_3}
	\mathop{\arg\min}_{\theta}(\sum(X − f (X,\theta))^2)
\end{equation}

通过约束网络中的中间层维度,从而可以通过提取用于聚类的中间稀疏编码,来获得原始数据的非线性降维。在这种情况下,使用相似的调制信号,可以由相似的卷积核和特征图来表示,因此,他们分布在该压缩空间的相近区域中。自编码器中的卷积层具有时移不变性以及受约束的参数搜索空间(相对于全连接层),因此非常适合于无线电时间序列信号表示。我们使用dropout[13]并在输入层加入噪声[7]对网络进行正则化,来增强模型的泛化能力。图\ref{fig_3_2}显示了我们的卷积自动编码器使用的体系结构。\par

\begin{figure}[!h]
	\centering
	\includegraphics[scale=0.3]{figures/chapter_3/CAE}
	\caption{自编码器}	\label{fig_3_2}
\end{figure}

在自编码器的训练过程中,我们尽量减少信号重构的均方误差(MSE),但由于我们的主要目标是获得原始信号的聚类稀疏表示,因此我们对重构误差作出简化假设:限制重构误差最小的情况下尽量降低隐藏层的维度。然而, 由于很难确定重构误差的最小值,所以,我们只能人为的指定隐层的维度来确定我们的稀疏表示维度,在维度确定的情况下调整参数使重构误差尽量小。图3显示了两个训练样本,2x128输入样本,1x30稀疏表示的特征,以及2x128输出重构的样本。我们使用RMSProp [11]和Adam [12]梯度下降求解器进行优化,两者都获得了相似的结果。\par

\begin{figure}[!h]
	\centering
	\includegraphics[scale=0.2]{figures/chapter_3/examples_cae}
	\caption{自编码器}	\label{fig_3_4}
\end{figure} 
通过上图,我们可以发现,卷积自编码器可以很好的复现原始信号。我们提取的特征可以很好的表征原始数据。这给出了卷积自编码器的学习表征能力的一些基本体现。\par。。。

为了可视化我们学到的卷及特征并对这些特征的类可分性进行展示,我们将数据的低维卷积特征,利用t-分布的随机相邻嵌入(t-SNE)[6]算法在二维流型上展示。在二维流型中相近的样本其在我们获得的低维空间中分布在相近的区域。因此,我们可以通过观察不同类别的数据样本在t-SNE可视化之后的二维流型上的分布,来反映样本无监督表示的类可分性。\par

我们从每一类样本中随机采样100个样本,将其通过训练的卷积自编码器获得其低维无监督表示,并利用t-SNE映射到二维流型,最终的效果如图X:

\begin{figure}[!h]
	\centering
	\includegraphics[scale=0.2]{figures/chapter_3/cae_fea}
	\caption{自编码器}	\label{fig_3_5}
\end{figure}

在这种情况下,我们看到几个类如WBFM,AMDSB, AM-SSB和QPSK已经形成了独立的,大部分可分离的簇,可以利用DBSCAN等聚类方法形成单独的类别;而其他类则会出现类别混淆,并且难以通过聚类方法分离类别簇。 尽管我们的无监督表示类可分性效果不是很好,但考虑到这些特征从来没有被训练用来区分不同类别的样本,我们就已经获得了数据一定程度的类可分性,这已经算是一个可以接受的结果了。 \par 

\subsection{调制信号的监督引导稀疏表示}

当我们具有一部分监督数据的时候,我们也可以使用监督训练时学习到的判别特征生成一个稀疏表示空间。利用TIM Shead的工作[14]中,我们利有标记样本以有监督的方式训练卷积神经网络。在训练好分类网络以后,我们移除最后的softmax层,保留剩余的这一部分网络。这样,在样本经过训练好的网络,最后隐层输出的特征即为原始数据的稀疏表示。图X中显示了我们利用监督方式训练网络,并获取监督引导特征空间,获取数据有监督的稀疏表示。\par

\begin{figure}[!h]
	\centering
	\includegraphics[scale=0.2]{figures/chapter_3/surprised_fea}
	\caption{自编码器}	\label{fig_3_6}
\end{figure}

我们从每一类样本中随机采样100个样本,通过图X的网络将其映射到监督引导特征空间,并利用t-SNE映射到二维流型,最终的效果如图X:

\begin{figure}[!h]
	\centering
	\includegraphics[scale=0.2]{figures/chapter_3/surprised_fea}
	\caption{自编码器}
\end{figure}

在这种情况下,我们几乎科技把每个调制类别的样本在二维流型中利用聚类算法分开。当然,其中也有一部分的数据是混淆的,比如类别X中也有部分样本散落到类别X中。\par
可能是因为在获取监督引导特征空间时,我们的目标是正确区分不同的调制类别,因此我们获取的监督引导特征对于不同类别的样本是有一定的区分度的,即不同调制类别的样本分布在特征空间的不同区域,这就表现为在t-SNE之后不同类别样本分布在二维流型的不同区域。\par

以这种方式形成的特征利用和提取可用的专业策划标签,但是在许多情况下,它们也推广并提供了在没有类别标签的特征空间中分隔额外类别的能力。因此,我们将其视为一种以监督方式进行稀疏特征学习的引导方法,并推广到包含额外的未标记训练数据的大型数据集上的半监督式识别。\par


\section{基于CAE-CNN的无线信号调制识别}
我们训练几个候选神经网络。利用两个卷积层和两个密集的完全连接层(CNN和CNN2)的4层网络。除了单热输出层上的Softmax激活之外,层使用整流线性(ReLU)激活功能。我们使用这个网络深度,因为它大致相当于在MNIST等视觉领域中类似的简单数据集上运行良好的网络。\par 

\subsection{CAE-CNN网络架构}
正则化用于防止过度拟合。 CNN使用Dropout,卷积层权重上的kW k 2范数惩罚,鼓励最小能量基,以及第一密集层激活上的khk 1范数惩罚来鼓励解的稀疏性[5] [10]。 


\subsubsection{CAE-CNN算法框架}

\begin{figure}[!h]
	\centering
	\includegraphics[scale=0.3]{figures/chapter_3/cae_cnn_frame}
	\caption{自编码器}	\label{fig_3_2}
\end{figure}

CNN2只使用丢失,而DNN只使用丢失。使用分类交叉熵损失函数和Adam [15]求解器进行训练,似乎在我们的数据集上略胜过RMSProp [12]。我们在Keras [16]上运行在TensorFlow [19]上的网络训练和预测,在DIGITS Devbox上的NVIDIA Cuda [8]启用的Titan X GPU上运行。图3显示了CNN架构的一个例子.CNN2是相同的但是较大的,在层1和层2中包含256和80个过滤器,在层3中包含256个神经元。评估的DNN包含4个密度层,大小分别为512,256,128 ,和n级神经元。\par

\subsubsection{CAE-CNN算法}


\begin{algorithm}[ht]
	\caption{Adam算法}
	\label{alg:adam}
	\begin{algorithmic}
		\REQUIRE 步长 $\epsilon$ (建议默认为: $0.001$)
		\REQUIRE 矩估计的指数衰减速率, $\rho_1$ 和 $\rho_2$ 在区间 $[0, 1)$内。
		(建议默认为:分别为$0.9$ 和 $0.999$)
		\REQUIRE 用于数值稳定的小常数 $\delta$  (建议默认为: $10^{-8}$)
		\REQUIRE 初始参数 $\theta$
		\STATE 初始化一阶和二阶矩变量 $s = 0 $, $r = 0$
		\STATE 初始化\gls{time_step} $t=0$ 
		\WHILE{没有达到停止\gls{criterion}}
		\STATE 从\gls{training_set}中采包含$m$个样本$\{ x^{(1)},\dots, x^{(m)}\}$ 的\gls{minibatch},对应目标为$y^{(i)}$。
		\STATE 计算梯度:$g \leftarrow \frac{1}{m} \nabla_{\theta} \sum_i L(f(x^{(i)};\theta),y^{(i)})$ 
		\STATE $t \leftarrow t + 1$
		\STATE 更新有偏一阶矩估计: $s \leftarrow \rho_1 s + (1-\rho_1) g$
		\STATE 更新有偏二阶矩估计:$r \leftarrow \rho_2 r + (1-\rho_2)  g \odot g$
		\STATE 修正一阶矩的\gls{bias_sta}:$\hat{s} \leftarrow \frac{s}{1-\rho_1^t}$
		\STATE 修正二阶矩的\gls{bias_sta}:$\hat{r} \leftarrow \frac{r}{1-\rho_2^t}$
		\STATE 计算更新:$\Delta \theta = - \epsilon \frac{\hat{s}}{\sqrt{\hat{r}} + \delta}$ \ \  (逐元素应用操作)
		\STATE 应用更新:$\theta \leftarrow \theta + \Delta \theta$
		\ENDWHILE
	\end{algorithmic}
\end{algorithm}

\subsection{学习复杂度}
我们使用Adam求解器训练了大约23分钟的最高复杂度模型,批量大小为1024的样本训练集大约需要15秒。我们确实观察到一些过度拟合,尽管没有正规化,但验证损失确实 没有显着变化,我们保持最佳的验证损失模型进行评估。\par

绘制学习的特征有时可以让我们直觉了解网络正在学习的底层表示。 在这种情况下,我们在下面绘制卷积层1和卷积层2的滤波器权重。 在图5中,第一层,我们有64个1x3的过滤器。 在这种情况下,我们只需获得一组边缘和梯度检测器,它们在每个I和Q通道上进行操作。\par

在卷积层2中,如图6所示的权重,我们将这个第一层特征图组合成64×16×2×3较大的特征图,其包括在I和Q通道上同时出现的情况。 这些特征图与在包括2D学习边缘检测器和Gabor滤波器的图像转换网络的较低层处所看到的特征图看起来没有太大的不同。\par


\section{结果及分析}

\subsection{分类准确率与鲁棒性}
为了评估分类器的性能,我们看一下测试数据集的分类性能。我们训练了一个包含11个调制模块的大约1200万个复杂样本的语料库。这些分为128个样本的训练样本。我们使用大约96,000个示例进行培训,并使用64,000个示例进行测试和验证。 这些样本均匀分布在从-20dB到+ 20dB的SNR中,并被标记以便我们可以评估特定子集上的性能。\par
\begin{figure}[!h]
	\centering
	\includegraphics[scale=0.3]{figures/chapter_3/loss}
	\caption{自编码器}	\label{fig_3_2}
\end{figure}
对于我们最高的SNR情况下的CNN2(0.6)分类,我们在图8中显示了一个混淆矩阵。在+18dBSNR时,在混淆矩阵中我们有一个干净的对角线,可以看到我们剩下的差异是8PSK误分类为QPSK,WBFM误分类作为AM-DSB。这两个都可以在基础数据集中解释。由于QPSK星座点由8PSK点跨越,所以包含特定比特的8PSK符号从QPSK难以分辨。在WBFM /AM-DSB的情况下,模拟语音信号具有只有载波音调存在的静默时段,这使得这些示例不可见。因此,即使在这个数据集的高信噪比下,也不可能获得100%的准确度,并使得重新合理的混淆被合理地容忍。\par

\subsection{训练效率以及分类效率}

在训练之后,我们在测试数据集上的所有信噪比之间的分类准确率大致达到了87.4%,但要理解这个意义,我们必须检查这个分类精度如何在不同训练样本的SNR值之间进行分解,以及 它与现有的基于专家特征的分类器的性能进行比较。绘制测试集调制分类精度,作为每个分类器的示例信噪比的函数7。 实线表示直接在无线电时间序列数据上进行深度特征学习训练的分类器,而虚线表示使用前面描述的专家特征作为输入的分类器。 这种观点是检验结果的关键方法,因为在低信噪比影响范围和覆盖范围的性能,我们可以有效地使用分类器。 我们从具有大量丢失正则化(0.6)的大卷积神经网络(CNN2)中获得显着更好的低SNR分类准确性性能。 在低信噪比情况下,最佳CNN模型的性能比基于专家特征的系统的信噪比高2.5-5dB,而+ 5dB SNR性能相似。 这是一个显着的性能改进,可能至少是传感系统有效覆盖面积的两倍。\par

\begin{figure}[!h]
	\centering
	\includegraphics[scale=0.3]{figures/chapter_3/result}
	\caption{自编码器}	\label{fig_3_2}
\end{figure}

为了更好地理解性能如何随信噪比而变化,我们检查了不同信噪比级别的几个分类器的混淆矩阵。\par

在非常低的信噪比(-6dB)的情况下,在图9,10,11和12中,我们看到一个有趣的情况,其中±20%内的所有精度都在50%左右。在这种情况下,CNN2分类器上的清洁器对角线比其他3种情况显着更明显,在这个区域学习到的特征具有显着的性能优势。\par

现在,所有4个分类器的信噪比(0dB)略高但仍然很低,现在有一个明确的对角线,但是我们发现在8PSK情况下发生的非对角线误分类更少。\par

6 模型复杂性\par

许多无线电系统中的一个重要考虑因素是训练和分类运行时间,由于计算复杂性。 深度学习的一个普遍批评是对大量计算资源的需求,然而在本文中,我们的网络相对紧凑,数据集相对较小。 我们比较下面每个模型的训练和分类运行时间。 在图17中我们可以看到,我们的CNN模型确实需要大量的训练时间,但是比SVM训练案例所需要的时间要少。\par
\begin{figure}[!h]
	\centering
	\includegraphics[scale=0.5]{figures/chapter_3/train_time}
	\caption{自编码器}	\label{fig_3_2}
\end{figure}
\begin{figure}[!h]
	\centering
	\includegraphics[scale=0.5]{figures/chapter_3/classify_time}
	\caption{自编码器}	\label{fig_3_2}
\end{figure}

在图18中显示,使用Keras编译python的这个模型的分类时间比使用scikit-learn的最近邻和SVM模型的大多数其他模型显着更快。只有决策树和GaussianNB模型获得更快的分类运行时间。在这两种情况下,基于ConvNet的这种规模的这种数据集分类模型提出了一个有吸引力的选择这个任务时,分类性能考虑。\par

通过约束网络中的中间层宽度,从而可以以低误差重建原始全维例子,通过提取用于聚类的中间稀疏编码来获得非线性降维。在这种情况下,使用类似的基函数和脉冲形状的调制可以由类似形状的卷积滤波器和编码器和解码器内的中间特征映射来表示,使得它们存在于该压缩空间的相似区域中。自编码器中的卷积层由于它们在时间上的不变表示性质以及受约束的参数搜索空间相对于它们的完全连接层等价物而非常适合于无线电时间序列信号表示。我们使用丢失[13]和输入噪声[7]作为正则化技术来帮助改善我们在数据集上学习表示的泛化。图2显示了我们的卷积自动编码器使用的体系结构。\par

\section{本章小结}

虽然这些结果并不是现有的最好的基于专家特征的调制分类器的全面比较,但是它们证明了,与相对专业的被认为的方法相比,时间序列无线电信号数据上的盲卷积网络是可行的并且工作得很好。在图7中,我们比较了几种分类器策略的准确性和信噪比,并且认为对于低信噪比和短时间的示例(128个复杂采样),这代表了调制分类的最先进的精确度方法。这种方法有可能容易地扩展到额外的调制类别,并且应该被视为依赖于无线电发射器的稳健的低SNR分类的DSA和CR系统的有力候选。\par

我们的结果与当前最好的专家系统方法的合理近似相比较,但是由于在无线电领域新兴的机器学习领域不存在强大的竞争数据集,所以很难直接比较性能和当前的现有技术状态。我们希望在以后的工作中进一步评估这一点,并将特征学习和专家方法从目前的水平上进行改进。 CNN2网络体系结构上的性能改进是不可避免的,我们花费了一些努力来优化它,但并没有做到这一点。较大的过滤器,不同的体系结构和池化层可能会显着影响性能,但是在这项工作中没有充分考虑其适用性。许多附加技术可以应用于这个问题,包括引入附加通道引起的效应的不变性,例如膨胀,I / Q不平衡,相位偏移等等。空间变换网络[17]已经证明了学习图像数据的这种不变性的强大能力,并且可以作为一个有趣的候选者,使得能够改善对这些效应的不变性学习。序列模型和递归层[13]可能能够表示信号序列嵌入,并且在更长时间表示中几乎肯定会证明是有价值的,但是我们还没有完全调查这个区域。这个应用领域已经成熟,可以进一步研究和应用,这将大大影响无线信号处理和认知无线电领域的技术发展水平,并将其转向机器学习和数据驱动方法。\par

