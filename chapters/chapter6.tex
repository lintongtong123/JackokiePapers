\chapter{总结与展望}
\section{研究结论}
通信信号的调制识别是认知无线的重要组成部分,其在军用领域与民用领域都有重要的作用,需要我们去进一步研究实现调制的自动识别。
深度学习的发展对很多行业产生了深远的影响,其在通信中的应用也获得了广泛的关注;
如何将深度学习更好地用于通信领域中,需要我们做出进一步的探索。
目前深度学习应用于调制识别,在学术领域已经得到了很大的关注,并有前人做出了一定的贡献。
本文正是将深度学习中的算法应用于调制识别中,主要贡献如下:

首先,本文提出了一种CAE-CNN的深度学习框架。
将CNN分类器的卷积层部分作为CAE的编码层,
然后分别在其后接入解码器和类似于DNN的分类层,融合CAE的重构误差与CNN的分类误差,来提高分类准确率。
同时,本文提出了一种CAE-CNN框架的训练算法,将CAE与CNN以不同轮询次数的循环进行训练,
并在训练过程中以参量的形式,不断调整训练CAE与CNN的循环次数,不断降低重构误差与分类误差。
最终的分类结果我们发现,在信噪比为-4dB时,7种信号的分类准确率仍在85\%左右,这相对于传统的方法是具有一定的优势。\par

接下来,我们提出了一种传统特征与深度特征融合的框架。
传统的调制识别研究已经持续数十年,相对比较成熟,对于不同的调制方式有很多特征可以进行有效区分。
而我们利用深度学习也可以提取信号的一系列深度特征,而利用这些特征我们同样可以对信号进行分类。
我们分别利用Softmax、DNN、RF三种融合框架对传统特征与深度特征进行融合,构造新的分类器。
最终我们发现,使用RF框架的特征融合算法可以取得较高的准确率,在0dB时,7种信号的分类准确率可以达到90\%以上。
同时,我们发现特征归一化在DNN和Softmax这样的融合框架中可以有效提升分类准确率。\par

最后,们从网络底层研究网络超参数对调制识别性能的影响,
并从欠拟合与过拟合以及偏差与方差的角度来理解出现这些现象的原因。
CNN在调制识别时的性能似乎不受网络深度的限制,在卷积层数目为$3$时即可达到性能的最值,
此时再增大卷积层的深度,系统性能可能会发生下降。
第一层卷积核数目较小时,网络的性能几乎随着第二层卷积核数目的增大而减小。
而随着第一层卷积核增大,改变第二层卷积核数目,系统分类准确率变化很小,只有轻微的波动。
整体而言,卷积核宽度$W^{(1)}$和$W^{(2)}$较大的网络性能较好,
但是当卷积的宽度增大到$7$之后,分类性能较为稳定,仅仅发生微小的波动。\par

无线信号调制识别是一种不断发展的无线认知技术,更多复杂的调制方式可能会出现,
这给现有的调制识别系统带来了很大的挑战,需要我们进行进一步的研究。
深度学习的发展很大程度上提高了机器学习算法性能的上界,而其在调制识别领域得到的应用还相对较少,
需要我们进行包括但不限于算法融合、特征融合、数据融合、深度网络框架等相应的研究。\par

\section{未来展望}
本文的算法是在高斯白噪声条件下进行的,如何在瑞利、多径信道条件下,利用深度学习实现较好性能的调制信号识别,仍然需要继续深入研究。
随着数字调制技术的发展,新的、更复杂的调制方式不断出现,如何通过深度学习的方法,去发现位置类别的信号,将会是我们后续的研究点。 
随着深度学习的发展,各种网络框架层出不穷,网络的底层结构也有诸如LSTM、CNN、CLDNN、InceptionNet、GoogleNet等框架,
如何将这些框架应用到调制识别中,什么样的网络框架可以提高我们分类识别的性能,我们目前正在探索,这也将是我们后续研究的重点。
我们希望通过研究深度学习在调制识别中的应用,能够提高调制识别的性能,并促进深度学习在通信领域的发展。\par