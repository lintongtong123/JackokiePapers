\chapter{总结与展望}
\section{研究结论}
通信信号的调制识别是认知无线的重要组成部分,其在军用领域与民用领域都有重要的作用,需要我们去进一步研究实现调制的自动识别。
深度学习的发展对很多行业产生了深远的影响,其在通信中的应用也获得了广泛的关注;
如何将深度学习更好地用于通信领域中,需要我们做出进一步的探索。
目前深度学习应用于调制识别,在学术领域已经得到了很大的关注,并有前人做出了一定的贡献。
本文正是将深度学习中的算法应用于调制识别中,主要贡献如下:

首先,本文提出了一种CAE-CNN的深度学习框架。
将CNN分类器的卷积层部分作为CAE的编码层,
然后分别在其后接入解码器和类似于DNN的分类层,融合CAE的重构误差与CNN的分类误差,来提高分类准确率。
同时,本文提出了一种CAE-CNN框架的训练算法,将CAE与CNN以不同轮询次数的循环进行训练,
并在训练过程中以参量的形式,不断调整训练CAE与CNN的循环次数,不断降低重构误差与分类误差。
最终的分类结果我们发现,在信噪比为-4dB时,7种信号的分类准确率仍能够接近90\%,这相对于传统的方法是一个较大的改进。\par

接下来,我们提出了一种传统特征与深度特征融合的框架。
传统的调制识别研究已经持续数十年,相对比较成熟,对于不同的调制方式有很多特征可以进行有效区分。
而我们利用深度学习也可以提取信号的一系列深度特征,而利用这些特征我们同样可以对信号进行分类。
我们分别利用Softmax、DNN、RF三种融合框架对传统特征与深度特征进行融合,构造新的分类器。
最终我们发现,使用RF框架的特征融合算法可以取得较高的准确率,在0dB时,7种信号的分类准确率可以达到90\%以上。
同时,我们发现特征归一化在DNN和Softmax这样的融合框架中可以有效提升分类准确率。\par

最后,我们研究了网络超参数对调制识别准确率的影响。
固定卷积核的数目和大小,将卷积层数目作为参量变化,我们发现:
当卷积层只有一层时,分类准确率会受到影响;
但是当卷积层超过两层之后,增加卷积层并不能明显提升分类性能。
固定卷积层数与卷积核大小,调整卷积核的数目,我们发现:
当第一个卷积层中卷积核数目较小时,增加第二层卷积核的数目并不能带来性能的提升;
而当第一个卷积层中卷积核数目增大后,调整第二个卷积层的数目,对于模型的性能无法带来提升。
固定卷积层数目和卷积核的数目,改变卷积核大小,我们发现:
卷积核宽度为8左右时,算法的分类性能最强。

无线信号调制识别是一种不断发展的无线认知技术,更多复杂的调制方式可能会出现,这给现有的调制识别系统带来了很大的挑战,需要我们进行进一步的研究。深度学习的发展很大程度上提高了机器学习算法性能的上界,而其在调制识别领域得到的应用还相对较少,需要我们进行包括但不限于算法融合、特征融合、数据融合、深度网络框架等相应的研究。

\section{未来展望}
本文的算法是在高斯白噪声条件下进行的,如何在瑞利、多径信道条件下,利用深度学习实现较好性能的调制信号识别,仍然需要继续深入研究。
随着数字调制技术的发展,新的、更复杂的调制方式不断出现,如何通过深度学习的方法,去发现位置类别的信号,将会是我们后续的研究点。 
随着深度学习的发展,各种网络框架层出不穷,网络的底层结构也有诸如LSTM、CNN、CLDNN、InceptionNet、GoogleNet等框架,
如何将这些框架应用到调制识别中,什么样的网络框架可以提高我们分类识别的性能,我们目前正在探索,这也将是我们后续研究的重点。
我们希望通过研究深度学习在调制识别中的应用,能够提高调制识别的性能,并促进深度学习在通信领域的发展。\par