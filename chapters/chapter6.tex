\chapter{总结与展望}
\section{研究结论}
通信信号调制识别是通信系统中信号解调、信息提取和信号检测的关键技术,因此是一个很有研究价值的课题,近年来取得了很多成果,在电子对抗,软件无线电等领域都有了广泛的应用。在近一年的时间里,查阅了大量的文献,广泛学
习了通信信号调制识别领域的相关研究成果。相对于其它调制识别算法,由于小波变换对瞬时信息具有良好的检测性能,并且在进行通信信号调制类型识别时不需要任何的先验知识,可以在中频上直接对信号进行识别,因此本文针对基于小波变换的调制识别技术进行了深入的研究。下面是本文的主要工作: \par

提出了一套基于小波变换的通信信号调制识别算法。能够在低信噪比条件下直接对所接收到的各种中频数字信号进行识别,与其它同类调制类型算法相比具有较高的识别性能,有一定的实用价值。 对于类间识别,首先,针对传统的基于小波变换幅度信息的调制识别技术进行研究,分析其不足及在低信噪条件下识别率低的问题。然后,提出了一种利用信号包络方差和小波变换频率信息的类间识别算法:首先,利用信号包络的方差分类 MQAM 信号和 MFSK、MPSK 信号,然后采用最优尺度对信号进行小波变换,再提取小波变换相位信息,第二次采用小波变换从小波变换的相位信息中提取瞬时频率信息,根据所提取的频率信息统计方差特征值以区分 MFSK 信号和 MPSK信号。 \par

对于类内识别,针对传统小波变换算法利用相邻码元处的小波变换幅度信息进行 MPSK 信号识别和码元区间内小波变换幅度信息进行 MFSK 信号识别,在低信噪比条件下受噪声影响显著的问题,提出了利用小波变换后相邻码元对应采样
点的相位差信息对 MPSK 信号进行类内识别,利用小波变换从小波变换相位信息中提取频率信息进行 MFSK 信号的类内识别,经过仿真验证,本文方法在相同仿真条件下,具有较高的识别性能,且稳定性好,实现了低信噪比条件下的调制类型识别。由于不同的 QAM 信号瞬时功率均值归一化后其分布特征具有很大差别,本文利用该特点,对小波变换后的瞬时功率进行均值归一化,以识别不同的 QAM信号。 \par

通信信号调制识别是一项不断发展的技术,新的、更复杂的信号调制方式及通信应用的出现,为其提出了越来越多的挑战,使得该技术领域不断有新的研究成果涌现,具有巨大的潜在研究价值。基于小波变换的数字信号调制识别算法的研究已经比较普遍,但是不论在理论方面还是在应用方面都还存在一些值得研究的问题。由于作者在这方面的研究时间较短,因此还有许多相关的问题需要进一步更深入的研究: \par

在调制识别中,如何充分利用小波变换方法和其它支撑矢量机或神经网络方法的优点,将它们有机结合,进一步提高在各种信噪比下的正确识别率。 \par
\section{未来展望}
本文的算法只是在高斯白噪声条件下进行的,如何在瑞利、多径信道条件下实现调制信号的识别,仍然值得继续深入研究。 随着数字调制技术的发展,新的、更复杂的调制方式不断出现,进一步研究与其它新的调制信号的分类问题等。 
希望今后能把通信信号调制识别技术深入的研究下去,做出自己的贡献。\par