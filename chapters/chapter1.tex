
\chapter{绪论}
\section{研究背景}
通信的主要目的是传达信息。 无论是在传统的有线通信系统还是在无线通信系统中,由于信道传输特性的限制,基带信号不能直接通过信道传输,而需要调制,基带信号可以移动到 一定的频段进行有效传输。 同时,对基带信号的调制还可以使通信系统具有更高的通信速率并提高频谱利用率。\par

随着通信技术的不断发展,用户对于信息传输速率要求的不断提高,以及人们对于无线信号便利性的需求,通信信号调制方式由传统的数字调制发展到主流的数字调制方式,信道从有线信道到有线与无线信道混合组网,信号调制形式由简单到复杂等。模拟调制针对的是模拟信号,传统的模拟调制按照调制方式的不同,可以将其分为幅度调制(AM)、频率调制(FM)和相位调制(PM)等。数字调制针对的是数字信号,按照调制方式的不同可以分为幅移键控(ASK)、频移键控(FSK)、相移键控(PSK)和正交幅度调制(QAM)等。 \par

通信技术的迅速发展导致了各种通信系统的共存。 然而,这些通信系统的调制方式和干扰技术不同,对系统间通信造成很大的障碍。 各种调制技术也导致接收机类型大幅增加,于是人们开始寻求新的通信方法,希望能够基于相同的接收机平台接收最好的调制,以减少接收机类型的快速增长,并且软件 无线电技术是解决这个问题的有效方法。 自动识别通信信号调制模式是软件无线电技术必须解决的关键问题之一。\par

调制是区分不同类型的通信信号的重要特征。所谓的调制的自动识别意味着它可以在给定未知调制信息的接收信号的情况下确定通信信号的调制模式,并在没有人为干预的情况下估计相应的调制参数。在密集和复杂的多用户频谱环境中自动识别通信信号调制。无线电信号识别是优化频谱利用,识别和最小化干扰,实施频谱策略以及实施有效的无线电意识和协调系统的重要工具。调制自动识别技术在军队和政府部门的无线电管理中也起着重要作用。在军事领域,调制识别被广泛用作通信侦察,电子战和威胁分析中的智能信号分析和处理的关键技术[9]。在现代战争中,战场信息的传播主要依赖于无线电通信。侦察通讯信号是电子战的主要内容。在EW侦听和侦听接收机的设计中,获取接收到的通信信号的调制方式是侦听接收机的重要功能之一。为解调器正确选择解调算法提供参数依据,最终得到有用的情报信息。调制识别技术还有助于为电子战选择最佳干扰模式或干扰消除算法,保证友好通信,同时压制和摧毁敌方通信,达到电子战通信对抗的目的。\par

另外,在自适应调制系统中,传输信号的调制模式随信道状态而变化。 为了正确解调信号,接收端需要知道发送信号的调制信息。 发送信令的最简单的方式是将包含调制信息的控制信号发送到分组中的接收机,其可以由接收机解调以获得信号的调制。 但是,这种方法是基于有用信息的带宽。 如果发送的信号不包含调制参数信息,则可以减少开销。 这需要信号识别技术来检测接收到的信号并获得信号参数来解调它。\par

因此,通信信号调制自动识别技术具有非常广阔的应用前景,将成为现代战争通信和未来无线通信应用的重要组成部分。\par

深度学习(DL)是具有最高级分类能力的机器学习(ML)的一个分支[1]。 最近引起了人们的关注,并在各个领域得到了应用。 在ImageNet大规模视觉识别挑战赛(ILSVRC)的竞争中,许多研究团队提交了各种用于对象检测和图像分类的DL算法。 最近1000个识别对象的前五位数字的准确性达到了95%[2]。 在文献[3]中,作者指出可以对DL网络进行培训,以便利用卫星图像进行发展中国家的准确,廉价和可扩展的经济调查。 而且,生物信息学也可以从DL获益。 可以从DNA序列中找到连接,可以从X射线图像中识别手指关节,并且可以从EEG信号中检测到不受控制的事件[4]。\par

尽管DL在全世界蓬勃发展,但它在通信中的使用相对较少。 值得注意的是,一些传统的ML算法如支持向量机(SVM)和K近邻(KNN)已被用于媒体访问控制(MAC)协议识别[5]和调制分类[6]。 在通信系统中使用DL有几个优点。 首先,由于通信装置的数量多且通信数据速率高,因此可以在通信系统中使用DL所需的大量数据。 其次,DL可以自主提取特征,避免了手动特征选择这一具有挑战性的任务。 第三,由于DL正在迅速发展,除了调制分类之外,其他通信应用将具有相当大的潜力。\par

另外,它也是一种不断发展的技术。 新的和更复杂的信号调制方法和通信应用的出现给它们带来了越来越多的挑战,并且在这个技术领域出现了新的研究成果。 这一领域的巨大潜在研究价值在这方面的研究越来越受到国内外众多学者的关注。\par

本文主要研究深度学习在无线信号的调制识别领域的应用。\par


\section{研究意义}

无线通信的调制识别是正确实现无线通信解调和保证正常接收的重要组成部分。需要在无线通信系统中传输或传输的基带原始信息通常是频率非常低且分量很多的频谱分量。这些低频分量信号不适合在信道中直接传输,只能调制到不同的信道。传输,所以在接收端,只有正确确定调制方式,才能使用合理的解调方法接收来自发送方的信息。如果接收信号的调制方式的误差判断包括调制类型的判断或者载波频率和波特率的错误估计,则接收机的解调结果必须使原始发送的基带信息信号被销毁或部分销毁。最终,它无法正确解调或接收来自发件人的原始信息。\par

\subsection{调制识别的意义}
因此,在民用通信中,对于一些非合作性的通信,如政府无线电管理部口在日常的无线电管理中,需要对民用通信信号进行监控与管理,必要时需要侦听与拦截,在这种情况下,由于管理与被管理双方通信未必事先预定,通信的调制方式类型的识别、载波频率、被特率及其它调制参数的正确估计显得尤为重要。\par

在军事训练中的无线电对抗、电磁干扰与反干扰等,在检测到对方无线电信号或电磁信号的同时,需要对检测到的信号或电磁信息进行调制识别包括载玻频率与波特率的正确巧计,以便进一步侦听破译对方通信情报信息或者对对方进行高强度的无线电干扰或者电磁干扰,使对方无线通信受到阻碍甚至痛疾。\par

在认知无线领域,当前无线频谱根据具体业务的不同,相应划分为民用的广播电视、无线通信、丑星通信及军用的雷达与军事通信等不同频段。同时,为了避免相邻频段或频道的互相干扰,不同的通信业务工作频段相互隔离并且预留一定隔离频带,上下频段频谱利用的极不平衡,造成频谱资源的极大浪费。显然,频谱资源利用率的不平衡性W及当前有限的无线频谱资源和僵硬的频谱管理方法已经成为制约无线通信进一步发展的重要因素,因此,如何在现有无线频谱资源基础上,在频率、时间和空间等H维空间上如何有效分配频谱资源,提高频谱利用率,认知无线电\par

\subsection{深度学习与无线通信的结合}

无线电通信是一个特定的信号处理领域。 在这个领域中,调制信号分类的专家特征和决策标准的使用已被广泛使用。 经过多年的积累,它已经能够在特定情况下实现高识别率。 在ImageNet竞赛中,识别深度学习的能力超出了人类专家的能力,提高了传统机器学习算法的上限。 此外,通过差异,卷积和合并等一系列方法[10],我们还增强了模型应用的泛化能力,使得深度学习在实际生产中得到广泛应用。\par

深度神经网络已经在计算机视觉,自然语言处理和语音识别等领域实现了广泛的应用,并取得了很好的结果。然而,深度学习应用于图像处理[11]和语音识别[18]的趋势绝大多数是从数据学习特征而不是利用专家特征进行学习,这表明我们对于认知无线领域,需要重新考虑一下是否应该也进行相应的思路转变。同时,深度学习在通信领域的应用,也带来了新的机遇与变革,XXX等人利用深度学习进行半监督识别,有效的利用了无监督数据,XXX等人将深度学习应用于路由,提高了识别准确率等。\par

综上所述,如何有效地提取特征参数、采用不同的识别算法与分类器实现调制式的自动识别,在军用和民用通信有广泛的应用,是软件无线电、认知无线电、频谱感知等领域研究的基础。随着通信技术的发展,通信环境日益复杂,新的调制方式不断出现,影响识别效率的因素越来越多,如何在现有自动识别技术方法基础上不断创新、提高仍然是一项颇具挑战性的研究领域。\par


\section{无线信号调制识别的发展和研究现状}

早期调制识别的任务是由操作人员在仪器的帮助下完成的。通过观察和分析接收信号的时域波形和频谱形状,判断信号的调制方式,然后选择相应的解调器进行解调。然而,随着无线通信技术尤其是数字通信技术的快速发展,信号调制方式变得越来越复杂,而且这种方法没有功率[20]。 1969年4月,C.S.Waver [3]等人在斯坦福大学发表了第一篇关于通信信号调制识别的论文“使用模式识别技术的调制类型的自动分类”。此后,调制方式的自动识别引起了人们的广泛关注,各种技术出版物上出现了许多关于调制识别的论文,其中许多结果已逐渐应用到实际工作中。 \par

对于通信信号调制识别的研究,现有算法大致可以分为基于假设检验的最大似然法和基于特征提取的模式识别方法,以及基于深度学习的调制识别方法。大多数基于假设检验的最大似然类方法计算复杂度较高,对模型失配问题更为敏感,这大大限制了它们在实际通信环境中的应用;基于特征提取的模式识别方法它通常关注能量检测,并使用专家特征和判定标准来识别和分类特定的调制类型。在适当的条件下,它可以实现接近最佳的识别性能,并且其性能更稳健,因此得到更广泛的应用;深度学习其他领域取得的成果,深度学习可以与硬件自适应学习分类器相结合,具有良好的适应性,并且提高了识别的准确性和鲁棒性,因此已成为近年来调制识别的研究热点。\par

\subsection{基于似然比判决理论的方法}

似然比决策理论方法也被称为最大似然假设检验法。 其基本框架是使用概率论和复合假设检验理论来分析信号的统计特性,并根据成本函数最小化原则导出足够的统计信息进行分类。 ,然后将其与适当的阈值进行比较,完成调制识别的自动分类功能。\par

对于现有的发展,Kim和Polydoros [4]使用平均似然比检验来识别BPSK和QPSK信号。通过近似似然函数的级数展开,提出了近似对数相似度的准对数似然比函数。但是,为了降低功能的计算复杂度,当信噪比为0dB时的识别率达到100%。该方法还可以扩展到MPSK,QAM等多级调制方法的识别。 Schreyoegg [5]等人。假设QAM信号的幅度和相位彼此大致相互独立,并且使用对数似然函数来使用振幅和相位的联合概率密度函数来识别MQAM信号。 Wei和Mendel [6]使用复数符号序列的平均似然函数对QAM信号进行分类。当信噪比为5dB时,识别率达到100%,并分析了最大似然分类器的渐近行为。它表明最大似然法可以对任何有限星座点的调制进行分类。当可用符号的数量接近无穷大时,分类错误概率接近零。它还给出了参数与错误概率之间的关系。关系表示给定错误概率所需的SNR和符号的数量。巴黎将这种通用方法与繁忙均衡相结合,研究了多径信道中的调制识别问题[7]。 Baudant等人提出了瑞利衰落信道中FSK信号的调制分类方法[8]。似然函数来自被高斯白噪声污染的FSK信号的包络。得到最优分类器并得到有效解决。衰落信道中的FSK信号调制识别问题。\par
 
似然比决策理论算法的优点是在理论上保证贝叶斯最小误判惩罚准则下分类结果最优,通过理论分析可以得到分类性能曲线。但是他的同事有很多局限性:首先,他们需要比模式识别方法更多的先验知识。现有的似然比决策理论算法主要处理符号同步采样序列,这意味着要知道信号的载波频率,符号速率和符号时序;其次,未知参数的存在导致似然比分类充满了复杂的统计表达式。它有很大的计算量,很难实时处理。第三,似然比决策算法对模型失配和参数偏差更敏感,即鲁棒性较差。似然比函数分类通常被建模为高斯分布,并且诸如信噪比等参数是已知的。当实际信道噪声为非高斯时,或存在多路径影响,多信号干扰和SNR参数估计偏差时,似乎分类性能可能急剧下降。\par
 
\subsection{基于特征提取的统计机器学习方法}

统计机器学习方法基于统计机器学习理论。 基本框架是首先从信号中提取先前选择的特征,然后执行模式识别,包括特征提取子系统和模式识别子系统。 特征提取子系统主要从未处理信号中提取所需的特征分量,如瞬时频率,瞬时相位和瞬时幅度。 模式识别子系统的主要功能是通过特征子系统提取的特征分量来区分信号的调制类型。 \par

关于数字调制信号识别方法的最早的公开讨论是Liedtke [10]。 他提出了一种用于未知调制方法分类的通用分类器。 该算法可以识别2ASK,2FSK,BPSK,QPSK,8PSK和CW信号,只需要大致了解信号的载波频率和符号率。 Azzouz和ANDI提出了一种算法[11-15],它使用相位的非线性部分,相位的非线性部分的绝对值,归一化的瞬时振幅和频率等。调制的标准偏差 使用一系列阈值或神经网络分类器来确定参数以识别调制的类型。 后来研究人员在此基础上进行了更多研究 Wong和AK Nandi随后对方法进行了改进[16],增加了信号的统计参数和训练序列,并且在0dB%时识别率达到了98%。\par
高阶统计量是描述随机过程的高阶统计特性的数学工具,包括高阶矩和高阶累积量,以及高阶周期矩和循环累积量。 由于Reichert J [19]首次提出使用高阶统计量来识别2ASK,BPSK,2FSK,MSK信号,所以基于高阶统计量的信号识别方法发展迅速。 Ananthram Swami [20] [21]使用四阶累积量来识别BPSK,4ASK,16QAM和8PSK信号。 当信噪比为10d B时,识别率为95%。 Spooner CM [22]将基于累积量方法的前一种最大化方法打破为四阶例程,提出用6阶循环累积量识别信号,并获得良好的识别效果。 当SNR为9dB时,16QAM总和64QAM的识别率分别为81%和90%。 QPSK和16QAM信号的识别率分别为97%和100%。\par

基于星座的调制识别将调制识别问题转化为形状识别问题。它可以使用许多现有的图像处理和模式识别算法,适用于正交调制信号分类,如MPSK,MASK和MQAM信号。识别MFSK信号。 Bijian [25]详细分析了基于星座图特征的调制识别方法。通过使用模糊逻辑方法可以很好地恢复原始信号(即接收信号的星座)的特性。在最大后验概率贝叶斯准则下,对模板进行匹配和判断。在信噪比大于0dB的情况下,QPSK,8PSK和16QAM分类的正确率可以达到90%。 Donoho等人[26]使用星座点处的信号计数来执行调制分类。该算法简单易行。该算法在信噪比大于15dB的条件下是稳定的。\par

近年来,基于小波变换的多分辨率特征分析,调制参数估计和调制识别器方兴未艾。 1995年,K C Ho [2]首次使用小波变换来识别MFSK和MPSK信号。取得了一定的成果,证明了该方法的可行性。 1999年,梁红等人。 [28]使用IF信号的小波变换系数的方差来分类FSK,PSK和QAM信号。 Haa小波变换相位突变检测能力的主要用途是提取CNR5d B信号中信号的分类特征。识别率达到97.6%,可以很好地对16QAM,QPSK和4FSK信号进行分类。在2000年,基于提取信号的小波细节特征的KCHor,529]的条件下,计算小波变换在固定尺度下的幅度值,并根据过渡符号的峰值数量使用统计直方图,直流电平的数量。识别MFSK和MPSK信号,通过估计符号周期和同步时间,进一步提高低信噪比下的分类成功率。当CNR = 6dB时,当CNR = 1dB时,MPSK信号的识别率为90.2%。 MFSK信号的识别率为91.9%。\par
 
考察上面使用的各种分类特征,已知它是否基于时域频域中的调制信号的特征,高阶统计量,星座特征或小波变换,人们很难找到调制分类。必须为每个分类问题分别考虑共同特征和方法,并根据所需调制类型搜索特定方法和特征。另外,大多数现有的调制识别算法都假定信道是理想的高斯白噪声信道,并且只有少数算法研究了衰落信道。在实际应用中,无线信道的衰落现象不容忽视,多径效应使得传输信号间存在码间干扰,多普勒效应对信号产生寄生调频。在基于理想高斯白噪声信道环境的识别算法中,信噪比较高时识别性能较好。当信噪比较低时,算法估计的瞬时包络,相位和频率参数偏离了感兴趣的通信。由于信号的实际参数特征,使识别性能急剧下降,稳定性差,不能满足实际应用的需要[34]。此外,目前大多数识别算法都局限于简单的调制方法,如MPSK信号,MFSK信号等。只有少数算法研究了MQAM信号的识别[35]。\par

此外,基于特征的识别方法依赖于信号属性,特征和决策统计的先验知识来分离已知的调制,并且通常在简化的分析硬件,传播,无线电环境模型下导出。\par

\subsection{基于深度学习的无线调制识别方法}
深层神经网络是由一系列层组成的大函数近似,其中每层表示从输入到输出激活的一些变换,其基于具有一些学习权重的参数传递函数。每一层通常是已知的具有可调参数和非线性激活函数的线性函数,使得所得到的函数组成可以是高度非线性的[3]。在深度神经网络中的函数参数通常用梯度下降优化器根据在网络输出上计算的一些损失函数进行训练。对于多类分类任务,如调制识别,目标函数通常是分类交叉熵(两个概率分布之间的差异的量度)。利用深度学习进行分类,已被成功地用于多类视觉任务,如Imagenet数据集上的对象识别[7]。\par

最近,已有学者证明了有监督的无线电分类系统的原始数据特征学习的可行性[14]。在这种情况下,我们能够超越传统的基于专家决策统计的敏感性和准确性分类的显着差距。这是一个强有力的结果,对当前的解决方案提供了显着的性能改进,但它仍然完全依赖于监督学习和精心策划的培训数据。

半监督聚类。在现实世界中,特别是在无线电领域,我们面临着大量的未标记的示例数据可用于我们的传感器和不完整的地面真相的类标签知识。为了解决这个问题,XX研究了无线电识别学习的替代策略,这些策略对标记的训练数据的依赖程度较低,并且能够使用无标记或较少标记的例子来理解无线电信号,这可能极大地减轻了在这种机器学习系统供开发人员和维护人员使用,并允许系统识别新的信号并扩展以了解新环境。\par

在半监督学习[9] [8] [5中,我们试图分离和识别新的类,没有明确的类标签,这些类的例子允许我们专门学习特征来分离它们。为了解决这个问题,我们考虑将信号示例的维数降低到一个平滑的较小的空间,我们可以执行信号聚类。给定一个适当的维数降低,我们寻求一个空间,相同或相似类型的信号具有较低的距离,而不同类型的信号间隔较大的距离。理想情况下,在这样的空间中,相同或相似类型的例子形成离散且可分离的簇,彼此容易辨别。在这样的空间中信号类型的分类则成为识别聚类,将标签与每个聚类相关联(而不是每个例子,一个劳动密集度较低的任务)的问题,以及允许识别,分类和历史注释的问题新的班级和新的班级例子,即使没有标签的知识。\par

卷积神经网络(CNN)[8]是最流行的DL体系结构之一,被认为是调制分类。调制的信号被转换成网格状的拓扑数据,例如星座图的图像,以馈送CNN。着名的CNN模型AlexNet [9]被采纳和修改用于网络训练和精度测试。 \par

密集的多用户环境中的无线电发射器识别是优化频谱利用,识别和最小化干扰以及执行频谱策略的重要工具。 无线电数据是容易获得的并且易于从天线获得,但是标记和策划的数据通常很少,使得监督式学习策略在实践中困难且耗时。 我们证明,半监督学习技术可以用来扩大学习范围以外的监督数据集,允许识别和召回新的无线电信号通过使用基于非监督和监督方法的非线性特征学习和聚类方法的稀疏信号表示。\par


\section{本文主要工作及内容安排}
从现有的研究成果可以看出,虽然数字通信信号的调制识别算法有很多,但能够直接对中频信号进行识别的还比较少,大多都需要确知信号的载波频率、码元速率等参数,通过复杂的处理把信号转换为基带信号后再进行识别,否则会造成明显的性能损失或算法失效。而在实际的非协作通信中,接收机是不可能确知这些先验信息的,因此论文重点研究了基于小波变换直接对中频信号进行识别的算法,提出了一套利用小波变换的多种参数信息对数字通信信号进行类间识别和类内识别的分类方法。论文算法在低信噪比条件下具有较高的识别率。 \par

第一章 对通信信号调制识别技术的研究背景和研究现状进行了介绍。\par

第二章 首先介绍了现代数字通信信号的调制技术和小波变换理论,然后对数字通信信号的小波变换进行了分析,为后面基于小波变换的调制识别研究打下理论基础。 \par

第三章 对数字通信信号的类间识别进行研究。首先介绍了小波变换中最优小波尺度的选择方法,然后,针对传统基于小波变换幅度信息的类间识别算法在低信噪比下识别率低的问题,提出了一种基于信号包络方差和小波变换频率信息的类间识别算法,最后通过仿真验证了该算法在低信噪比下具有较高的识别性能。 \par

第四章 对 MPSK、MFSK 和 MQAM 信号的类内识别进行研究,分别提出了利用信号小波变换后相邻码元对应采样点的相位差信息识别 MPSK 信号,利用信号小波变换后的频率信息识别 MFSK 信号和利用信号小波变换后的归一化瞬时功率值分布特征识别 MQAM 信号的分类算法。并通过仿真验证论文算法在低信噪比下具有较高的识别性能。 \par

第五章 总结全文,并对今后的工作进行了展望。\par

第六章 总结全文,并对今后的工作进行了展望。\par
