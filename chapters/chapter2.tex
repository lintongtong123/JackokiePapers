
\chapter{无线信号调制识别以及深度学习理论}
\label{chap: mod_rec_deep_learning_theo}
\section{引言}
在动态频谱接入(DSA)中,感知周围的发射终端,以避免无线干扰并优化频谱分配,是无线认知的重要组成部分。 广播无线电、卫星信号、4G无线信号、雷达用户以及附近其他潜在无线电干扰源等信号具有不同的调制形式和特征,识别和区分这些信号是通信系统最基本的步骤。因此,需要我们对信号调制方式进行判别解调。本章主要讨论信号调制的基本概念以及深度学习的基本理论。

\section{调制识别}

\subsection{信道对调制信号的影响}
无线信道模型是对无线信道的抽象描述,它能很好地反映真实环境中的信号传输规律。
无线通信数据信息主要以电磁波为载体通过无线信道传输。 
由于无线信道的环境复杂多变,电波以不同的传输方式(直射,反射,散射等)到达接收点,使得接收到的信号与发射的信号不同。
因此,只有准确预测无线信号的无线传播特性,如路径损耗和相位延迟,才能为无线网络提供合理的设计,部署和管理策略。\par

信道效应具备不确定性,在通信系统中是不可逆的。
真实的通信系统在进行信号传输时会经历许多影响,这给恢复和表示原始信号带来了很大难度。
热噪声在接收器处产生相对平坦的高斯白噪声,其形成信号的底噪。
由于温度和半导体物理材料自身特性,发射器和接收器的特性可能产生波动,
从而引起振荡器偏移导致符号时序偏移,采样速率偏移,载波频率偏移和相位差等。
这些效应可能导致信道之间的时间移位,缩放,线性混合、旋转等效应,给信息传输稳定性带来不利影响。
最后,根据在接收机处发射信号的到达模式,信号经过实际信道可能会经历随机滤波,产生幅度,相位变化和多普勒频移。
这就是我们通常所说的多径衰落或频率选择性衰落,其主要发生在当信号的传播路线上出现建筑物、车辆等障碍物,
阻碍了信号的视距传播,造成信号在空间中的反射,发生时频特性的变化。\par

\subsection{信道建模}

无线信号的调制识别可以看作是一个N类的决策问题。其中,我们的输入是一个接收信号的复时间序列。
也就是说,我们以离散时间步长对无线电信号的同相和正交分量进行采样,以获得$2 \times N$的复数值向量。\par

\begin{equation}\label{sec:eqt2_1}
r(t) = s(t)*c + n(t)
\end{equation}

我们将接收信号用等式\ref{sec:eqt2_1}表示。
其中,将连续信号或一系列离散的时间序列信号,调制到具有变化的频率、相位、振幅、或多个变换的正弦波上,得到调制信号$s(t)$。
$c$是信号上的一些路径损耗或恒定增益项,$n(t)$是反映热噪声的加性高斯白噪声过程。
从工程的角度来看,这个简化的表达式在基于专家特征的决策统计方法中被广泛使用。\par

然而,实际的信道环境却比较复杂。发射信号$s(t)$,在传播过程中经历多个信道效应, 最后在接收端被接收为$r(t)$。
这些信道效应包括:时间延迟,尺度缩放,相位旋转,频率偏移,加性热噪声,信道脉冲响应,以及所有的随机时变过程等。 
这些效应对信号的作用可以近似表示成方程 \ref{sec:eqt2_2}:\par

\begin{equation}\label{sec:eqt2_2}
r(t) = e^{j*n_{Lo}(t)} \int_{\tau=0}^{\tau_{0}} s(n_{Clk}(t-\tau))h(\tau) + n_{Add}(t)
\end{equation}

方程\ref{sec:eqt2_2}考虑了许多对于模型来说很重要的现实世界的影响:
通过残留载波随机游走过程调制$n_{Lo}(t)$,通过残留时钟振荡器随机游走重采样$n_{Clk}$,
与时变的旋转非恒定幅度脉冲响应$h(t-∞)$卷积,以及加性噪声$n_{Add}(t)$(可能不是白噪声)。
每个都可能导致未知的时变误差。考虑到现实世界中存在的无线信道的影响时,会使我们的接收信号表示复杂化。\par

考虑到传播信道的复杂性,对专家特征提取并进行分类决策建模是很难的。
这通常会迫使我们简化假设,构建易于处理的如方程\ref{sec:eqt3_1} 所描述的基本模型;
然而,基本模型很难刻画复杂的信道特征,这样就造成了算法性能的上限较低,鲁棒性较差。
在本文中,我们主要关注包括所有上述影响的模拟传播环境中的实测数据,利用数据反映信道本身的特征,
而不是从理论上的进行信道建模指纹特征提取等。\par

\section{神经网络概述}

\subsection{神经元概述}
神经网络(Neural Networks,NN),是一种模仿人类大脑神经行为,进行分布式并行信息处理的算法数学模型。
以监督学习为例,假设我们有训练样本集$(x(^i),y(^i))$,
那么神经网络算法能够提供一种复杂且非线性的假设模型$h_{W,b}(x)$ ,它具有参数$W, b$,
可以以此参数来拟合我们的数据。图\ref{fig_2_1}即是“神经元”的图示:\par
\begin{figure}[htbp]
	\centering
	% Requires \usepackage{graphicx}
	\includegraphics[scale=0.9]{figures/chapter_2/fig_2_1.png}\\
	\caption{神经元示意图}\label{fig_2_1}
\end{figure}
此“神经元”是一个以$x_1, x_2, ..., x_n$ 及截距$b$ 为输入值的运算单元,其输出为:
\begin{equation}
	h_{W,b}(x) = f(W^Tx) = f(\sum_{i=1}^n w_{i}x_i +b)
\end{equation} 
其中函数$f(z)$被称为“激活函数”,常用的激活函数有$ReLU、Sigmoid、tanh$等。在本论文中,我们一般选用$ReLU$函数作为激活函数$f(\cdot)$:
\begin{equation}
	f(z) = 
	\begin{cases}
		1 & z <=0\\
		z & z > 0
	\end{cases}
\end{equation}
可以看出,$ReLU$激活函数在输入值小于零时输出为0,在大于0时为一条经过原点的直线,其相较于$Sigmoid$激活函数而言,具有如下优点:单侧抑制 ,相对较宽的兴奋边界 ,并且具备稀疏激活性,利于梯度的传播。\par
图\ref{sec:fig_2_2}和图\ref{sec:fig_2_3}分别是$ReLU$及$Sigmoid$的函数图像:
\begin{figure}[htbp]
	\centering
	\begin{minipage}[t]{0.48\textwidth}
		\centering
		\includegraphics[width=8cm]{figures/chapter_2/fig_2_2.png}
		\caption{$ReLU$函数}\label{sec:fig_2_2}
	\end{minipage}
	\begin{minipage}[t]{0.48\textwidth}
		\centering
		\includegraphics[width=8cm]{figures/chapter_2/fig_2_3.png}
		\caption{$Sigmoid$函数}\label{sec:fig_2_3}
	\end{minipage}
\end{figure}

\subsection{前馈神经网络}
前馈神经网络就是将许多个单一“神经元”联结在一起,这样,一个“神经元”的输出就可以是另一个“神经元”的输入。比如,图\ref{sec:fig_2_4}就是一个简单的神经网络:\par
\begin{figure}
	\centering
	\includegraphics[scale=0.7]{figures/chapter_2/fig_2_4}\label{sec:fig_2_4}
	\caption{前馈神经网络}\label{fig_2_4}
\end{figure}
我们使用圆圈来表示神经网络的输入,标上“$b$”的圆圈被称为偏置节点,也就是截距项。神经网络最左边的一层叫做输入层,最右的一层叫做输出层(本例中,输出层只有一个节点)。中间所有节点组成的一层叫做隐藏层,因为我们无法在训练样本集中观测到它们的输出值。同时可以看到,以上神经网络的例子中有$n(n=3)$个输入单元(偏置单元不计在内),$n(n=3)$个隐藏单元,以及一个输出单元。\par

 我们用$l_{*}$来表示网络的层数,本例中 ,我们将第$i$层记为$l_i$ ,于是$l_{0}$表示输入层,$l_L$表示输出层 。
利用$a^{(l)}_i$表示第$l$层第$i$单元的激活值(输出值)。当$l=1$时,$a^{(1)}_i = x_i$,也就是第$i$个输入值(输入值的第$i$个特征)。对于给定参数集合$W,b$,我们的神经网络就可以按照函数$h_{W,b}(x)$来计算输出结果。\par
 
我们用$z^{(l)}_i$表示第$l$层第$i$单元输入加权和(包括偏置单元)。
上面的计算步骤叫作前向传播。
之前我们用$a^{(1)} = x$表示输入层的激活值,那么给定第$l$层的激活值$a^{(l)}$后,第$l+1$层的激活值$a^{(l+1)}$就可以按照下面步骤计算得到:
\begin{align}
	z^{(l+1)} &= W^{(l)} a^{(l)} + b^{(l)}=\sum_{j=l}^n W^{(l)}_{ij} x_j + b^{(l)}_i\\
	a^{(l+1)} &= f(z^{(l+1)})
\end{align}
将参数矩阵化,使用矩阵-向量运算方式,我们就可以利用线性代数的优势对神经网络进行快速求解。\par

以上我们讨论了一种单隐层的神经网络,我们也可以构建另一种复杂结构的神经网络
(这里结构指的是神经元之间的联接模式),也就是包含多个隐层的神经网络。
最常见的一个例子是$n_l$层的神经网络,第$1$ 层是输入层,第$n_l$层是输出层,
中间的每个层$l$与层$l+1$紧密相联。这种模式下,要计算神经网络的输出结果,
我们可以按照之前描述的等式,按部就班,进行前向传播,逐一计算第$L_2$层的所有激活值,
然后是第$L_3$层的激活值,以此类推,直到第$L_{n_l}$ 层。
这样我们就构建了一个神经网络,由于这种神经元之间的连接没有闭环或回路,因此我们称之为前馈神经网络。\par

\subsection{反向传播算法}
假设我们有一个固定的样本集$\{ (x^{(1)}, y^{(1)})$, $\ldots$, $(x^{(m)}$, $y^{(m)}) \}$,它包含$m$个样本了,那么我们可以用批量梯度下降法来求解神经网络。具体来说,对于单个样本$(x,y)$,我们假设其损失函数$J(W,b; x,y)$为方差代价函数,则有该样本的损失函数$J(W,b; x,y)$:
\begin{align}
	J(W,b; x,y) = \frac{1}{2} \left\| h_{W,b}(x) - y \right\|^2.
\end{align}
如果训练集包含$m$个样本,那么所有样本的整体损失函数$J(W,b)$为:
\begin{equation}
	\begin{aligned}
	J(W,b)
	&= \left[ \frac{1}{m} \sum_{i=1}^m J(W,b;x^{(i)},y^{(i)}) \right]
	+ \frac{\lambda}{2} \sum_{l=1}^{n_l-1} \; \sum_{i=1}^{s_l} \; \sum_{j=1}^{s_{l+1}} \left( W^{(l)}_{ji} \right)^2
	\\
	&= \left[ \frac{1}{m} \sum_{i=1}^m \left( \frac{1}{2} \left\| h_{W,b}(x^{(i)}) - y^{(i)} \right\|^2 \right) \right]
	+ \frac{\lambda}{2} \sum_{l=1}^{n_l-1} \; \sum_{i=1}^{s_l} \; \sum_{j=1}^{s_{l+1}} \left( W^{(l)}_{ji} \right)^2
	\end{aligned}
\end{equation}
以上关于$J(W,b)$定义中的第一项是一个均方差项,第二项是一个正则化项(也叫权重衰减项),
其目的是减小权重的绝对值,降低过拟合现象出现的概率。
权重衰减系数$\lambda$用于平衡公式中两项的相对重要性。
以上的代价函数经常被用于分类和回归问题。
在二分类问题中,如果我们使用$sigmoid$函数作为激活函数,由于$sigmoid$激活函数的值域为$[0,1]$,
那么可以使用$y=0$或$y=1$,来代表两种类别的标签;
如果我们使用双曲正切激活函数$tanh$,那么可以选用$-1$和$+1$ 作为类别标签)。\par

反向传播算法的思路如下:给定一个样例$(x,y)$,我们首先进行“前向传播”运算,计算出网络中所有的激活值,包括$h_{W,b}(x)$的输出值。之后,针对第$l$层的每一个节点$i$,我们计算出其“残差”$\delta^{(l)}_i$,此残差表明了该节点对最终输出值的误差产生了多少影响。对于最终的输出节点,我们可以直接算出网络产生的激活值与实际值之间的差距,我们将这个差距定义为$\delta^{(n_l)}_i$(第$n_l$层表示输出层)。对于隐藏单元我们如何处理呢?我们将基于节点残差的加权平均值计算$\delta^{(l)}_i$,这些节点以$a^{(l)}_i$作为输入。反向传播算法可表示为以下几个步骤:\par
进行前馈传导计算,利用前向传导公式,得到$L_2, L_3$, $\ldots$直到输出层$L_{n_l}$的激活值。
对输出层(第$n_l$层),计算:
\begin{align}
	\delta^{(n_l)}
	= - (y - a^{(n_l)}) \bullet f'(z^{(n_l)})
\end{align}
对于$l = n_l-1, n_l-2, n_l-3$, $\ldots$, $2$的各层,计算:
\begin{align}
	\delta^{(l)} = \left((W^{(l)})^T \delta^{(l+1)}\right) \bullet f'(z^{(l)})
\end{align}
计算最终需要的偏导数值:
\begin{align}
	\nabla_{W^{(l)}} J(W,b;x,y) &= \delta^{(l+1)} (a^{(l)})^T, \\
	\nabla_{b^{(l)}} J(W,b;x,y) &= \delta^{(l+1)}.
\end{align}
实现中应注意:在以上的第2步和第3步中,我们需要为每一个$i$值计算其$f'(z^{(l)}_i)$。假设$f(z)$是sigmoid函数,并且我们已经在前向传导运算中得到了$a^{(l)}_i$。
最后,我们将对梯度下降算法做个全面总结。在下面的伪代码中,$\Delta W^{(l)}$是一个与矩阵$W^{(l)}$维度相同的矩阵,$\Delta b^{(l)}$是一个与$b^{(l)}$维度相同的向量。注意这里“$\Delta W^{(l)}$”是一个矩阵,而不是“$ \Delta$与$W^{(l)}$相乘”。下面,我们实现批量梯度下降法中的一次迭代:\par

对于所有$l$,令$\Delta W^{(l)} := 0$, $\Delta b^{(l)} := 0 $(设置为全零矩阵或全零向量)
对于$i = 1$到$m$,
使用反向传播算法计算$\nabla_{W^{(l)}} J(W,b;x,y)$和$\nabla_{b^{(l)}} J(W,b;x,y)$。
计算$\Delta W^{(l)} := \Delta W^{(l)} + \nabla_{W^{(l)}} J(W,b;x,y)$。
计算$\Delta b^{(l)} := \Delta b^{(l)} + \nabla_{b^{(l)}} J(W,b;x,y)$。
更新权重参数:
\begin{align}
	W^{(l)} &= W^{(l)} - \alpha \left[ \left(\frac{1}{m} \Delta W^{(l)} \right) + \lambda W^{(l)}\right] \\
	b^{(l)} &= b^{(l)} - \alpha \left[\frac{1}{m} \Delta b^{(l)}\right]
\end{align}
接下来,我们可以对梯度下降法算法不断迭代,减小损失函数$J(W,b)$的值,直到达到一定阈值或者终止条件,进而求解神经网络。

\section{卷积神经网络}
卷积网络(Convolutional Neural Network, CNN),也叫做卷积神经网络,是一种专门用来处理具有类似网格结构的数据的神经网络,
是指那些至少在网络的某一层中使用卷积运算来替代一般的矩阵乘法运算的神经网络。
例如时间序列数据(可以认为是在时间轴上有规律地采样形成的一维网格)和图像数据(可以看作是二维的像素网格)。
卷积网络在诸多应用领域都有很好的表现。\par

\subsection{卷积运算}
\label{sec:the_convolution_operation}
卷积是一种对两个实变函数的线性运算,通常用星号表示:
\begin{equation}
s(t) = (x*w)(t) = \int x(\tau)w(t-\tau)d\tau
\end{equation}
在卷积网络的术语中,卷积的第一个参数(在这个例子中,函数$x$)通常叫做输入,第二个参数(函数$w$)叫核函数。
输出$ (x*w)$有时被称作特征映射。
如果我们假设$x$和$w$都定义在整数时刻$t$上,则卷积的离散形式:
\begin{equation}
s(t) = (x*w)(t) = \sum_{\tau = -\infty}^{\infty} x(\tau)w(t\tau).
\end{equation}
在机器学习应用中,输入通常是由样本的多维参数值组成的高维数组,而核通常是由优化算法对模型参数进行学习得到的多维数组。
通常我们称之为做张量。
因为在输入与核中的每一个元素都必须明确地分开存储,我们通常假设在存储了数值的有限点集以外,这些函数的其他值都为零。
这意味着在实际操作中,我们可以通过对有限个数组元素的求和来实现卷积中的无限求和。
在处理图像数据时,我们经常一次在多个维度上进行卷积运算。如果把一张二维的图像$I$作为输入,同时使用使用一个二维的核$K$,则有:
\begin{equation}
S(i,j) = (I*K)(i,j) = \sum_m \sum_n I(m,n) K(i-m, j-n).
\end{equation}
卷积是可交换的(commutative),我们可以等价地写作:
\begin{equation}
S(i, j) = (K*I)(i,j) = \sum_m \sum_n I(i-m, j-n) K(m, n).
\end{equation}
卷积运算可交换性的出现是因为我们将核相对输入进行了翻转,从$m$增大的角度来看,输入的索引在增大,但是核的索引在减小。
我们将核翻转的唯一目的是实现可交换性。\par
\subsection{卷积特性}
\subsubsection{卷积是一种无限强的先验}
先验被认为是强或者弱取决于先验中概率密度的集中程度。
弱先验具有较高的熵值,例如方差很大的高斯分布。这样的先验允许数据对于参数的改变具有或多或少的自由性。
强先验具有较低的熵值,例如方差很小的高斯分布。这样的先验在决定参数最终取值时起着更加积极的作用。
一个无限强的先验需要对一些参数的概率置零并且完全禁止对这些参数赋值,无论数据对于这些参数的值给出了多大的支持。\par
我们可以把卷积网络类比成全连接网络,但对于这个全连接网络的权重有一个无限强的先验。
这个无限强的先验是说一个隐藏单元的权重必须和它邻居的权重相同,但可以在空间上移动。
这个先验也要求除了那些处在隐藏单元的小的空间连续的接受域内的权重以外,其余的权重都为零。\par
总之,我们可以把卷积的使用当作是对网络中一层的参数引入了一个无限强的先验概率分布。
这个先验说明了该层应该学得的函数只包含局部连接关系并且对平移具有等变性。
\subsubsection{卷积性质}
卷积运算通过三个重要的思想来帮助改进机器学习系统:稀疏交互、参数共享、等变表示。
传统的神经网络使用矩阵乘法来建立输入与输出的连接关系。
对于卷积,参数共享的特殊形式使得神经网络层具有对平移等变的性质。
如果一个函数满足输入改变,输出也以同样的方式改变这一性质,我们就说它是等变(equivariant)的。
特别地,如果函数$f(x)$与$g(x)$满足$f(g(x))= g(f(x))$,我们就说$f(x)$对于变换$g$具有等变性。
对于卷积来说,如果令$g$是输入的任意平移函数,那么卷积函数对于$g$具有等变性。
当处理时间序列数据时,这意味着通过卷积可以得到一个由输入中出现不同特征的时刻所组成的时间轴。
如果我们把输入中的一个事件向后延时,在输出中仍然会有完全相同的表示,只是时间延后了。
而这也正好对应到我们无线信号中的时移不变性。因此,我们可以很好的将\label{con_net}应用到调制信号识别中。\par
\subsubsection{卷积特征学习}
通常,卷积神经网络训练中最耗时的部分是特征学习。 
通常输出层的计算代价相对不高,因为在通过若干层卷积池化之后,输入到该层的特征的数量较小。
当使用梯度下降执行有监督训练时,每一次进行梯度计算需要完整地运行整个网络的前向传播和反向传播,
以使残差从输出层反馈到输入层。
利用那些无监督方式训练得到的特征是一种常用的减少卷积网络训练成本的方法。\par

有三种常用的基本策略可以不通过监督训练而得到卷积核。
其中一种是简单地随机初始化它们。
另一种是手动设计它们,例如设置每个核在一个特定的方向或尺度来检测边缘。
最后,可以使用无监督的方法来学习卷积核。\par

使用无监督的标准来学习特征,我们可以认为是将CNN分成卷基层与分类层两部分,
网络允许我们将卷积层的特征参数确定与随后的分类层的参数学习相分离。
这样,便可以只需提取一次全部训练集的特征,作为随后分类层的新的训练集。
假设最后一层类似逻辑回归(多分类问题可以认为是Softmax回归)或者支持向量机,
那么分类层的学习通常是一种凸优化问题。\par
随机滤波器经常在卷积神经网络中表现得很好~\cite{Jarrett-ICCV2009-small,Saxe-ICML2011,pinto2011scaling,cox2011beyond}。
\cite{Saxe-ICML2011}~说明,由卷积和随后的池化组成的层,当赋予随机权重时,自然地变得具有频率选择性和平移不变性。
他们认为这提供了一种低时间成本的方法来选择卷积网络的结构:首先通过仅训练最后一层来评估几个卷积网络结构的性能,然后选择最好的结构并使用相对耗时的方法来训练整个网络。
一个折中方法是学习特征,但是使用那种不需要在每个梯度计算步骤中都进行完整的前向和反向传播的方法。
与DBM和多层感知机一样,我们可以使用贪心逐层预训练来对网络进行学习。
首先,单独训练第一层;
然后一次性地从第一层提取所有特征,并用这些特征单独训练第二层;
以此类推,直到网络能够达到我们期望的性能。

\section{卷积自编码器}
\subsection{自编码器}
自动编码器是一种无监督的学习算法,其优化目标是通过一些较低的中间隐层维度,使用均方误差(MSE)等损失函数,
最小化网络输出的重构误差。\par
自编码器是一种尽可能还原原始输入信号的神经网络,主要用于数据的降维或者特征的提取。
为了还原输入信号,编码后的数据必须保留原始数据的主要特征。
将原始的训练数据集样本作为自编码器的输入,如图\ref{sec:fig_2_5}所示,自编码器主要由两部分组成:
一个由函数 表示的编码器,以及一个生成重构信号的解码器 。
它内部有一个隐藏层 ,可以产生原始数据编码后的低维表示。
\begin{figure}[!h]
	\centering
	\includegraphics[scale=0.7]{figures/chapter_2/fig_2_5}
	\caption{自编码器}	\label{sec:fig_2_5}
\end{figure}
图\ref{sec:fig_2_5}中,从输入层$X$到隐层$h$再到输出层$\hat{X}$的网络就是自编码器模型,
它由编码器(Encoder)和解码器(Decoder)两部分组成,本质上都是对输入信号做某种非线性变换。
编码器将输入信号$X$变换成编码信号$h$,而解码器将编码信号$h$转换成输出信号$\hat{X}$。即:
\begin{equation}
	\begin{gathered}
		h=f(X)
		\\
		\hat{X}=g(h)=g(f(X))
	\end{gathered}
\end{equation}
通常,自编码器利用反向传播算法,将误差进行反向传播,并使用随机梯度下降(SGD)算法等,以找到接近等式\ref{sec:eqt2_3}中的最佳网络参数。
\begin{equation}\label{sec:eqt2_3}
\mathop{\arg\min}_{\theta} \sum(X − \hat{X})^2
\end{equation}
\subsection{卷积自编码器}
卷积自编码器是自编码器中的一种,主要是在编码器中加入了卷积层,使得网络可以学习到数据的卷积特征。
编码器主要是由卷积层和池化层较差组成的神经网络。卷积的作用相当于一个滤波器,而池化则是提取不变特征。
CNN和CAE之间最主要的区别在于前者是进行端到端的学习滤波器,并且将提取的特征进行组合从而用来分类。
事实上,CNN通常被称为是一种监督学习;而CAE则通常被用来训练从输入数据中提取特征,从而重构输入数据。\par

CAE中的卷积层具有时移不变性以及受约束的参数搜索空间(相对于全连接层),因此非常适合于无线电时间序列信号表示。
我们使用dropout[13]并在输入层加入噪声[7]对网络进行正则化,来增强模型的泛化能力。\par

\section{长短期记忆网络}
\label{sec:lstm}
\label{chap:sequence_modeling_recurrent_and_recursive_nets}
\gls{RNN}或~\glssymbol{RNN}是一类用于处理序列数据的\gls{NN}。
就像\gls{convolutional_network}是专门用于处理网格化数据的\gls{NN},\gls{RNN}是专门用于处理序列$x^{(1)}, \dots, x^{(\tau)}$的\gls{NN}。

正如\gls{convolutional_network}可以很容易地扩展到具有很大宽度和高度的图像,以及处理大小可变的图像,\gls{recurrent_network}可以扩展到更长的序列(比不基于序列的特化网络长得多)。
大多数\gls{recurrent_network}也能处理可变长度的序列。

\begin{figure}[!h]
	\centering
	\includegraphics[scale=0.7]{figures/chapter_2/fig_2_6.png}
	\caption{\glssymbol{LSTM}~\gls{recurrent_network}``细胞''的框图。
		细胞彼此循环连接,代替一般\gls{recurrent_network}中普通的\gls{hidden_unit}。
		这里使用常规的人工神经元计算输入特征。
		如果sigmoid输入门允许,它的值可以累加到状态。
		状态单元具有线性自循环,其权重由\gls{forget_gate}控制。
		细胞的输出可以被输出门关闭。
		所有\gls{gated}单元都具有sigmoid非线性,而输入单元可具有任意的压缩非线性。
		状态单元也可以用作\gls{gated}单元的额外输入。
		黑色方块表示单个\gls{time_step}的延迟。
	}\label{fig:chapter2_lstm_cell}
\end{figure}

\glssymbol{LSTM}~块如\figref{fig:chapter2_lstm_cell}所示。
在浅\gls{recurrent_network}的架构下,相应的\gls{forward_propagation}公式如下。
更深的架构也被成功应用\citep{Graves-et-al-ICASSP2013,Pascanu-et-al-ICLR2014}。
\glssymbol{LSTM}~\gls{recurrent_network}除了外部的~\glssymbol{RNN}~循环外,还具有内部的``\glssymbol{LSTM}~细胞''循环(自环),因此~\glssymbol{LSTM}~不是简单地向输入和循环单元的仿射变换之后施加一个逐元素的非线性。
与普通的\gls{recurrent_network}类似,每个单元有相同的输入和输出,但也有更多的参数和控制信息流动的\gls{gated}单元系统。

最重要的组成部分是状态单元$s_i^{(t)}$,与前一节讨论的\gls{leaky_unit}有类似的线性自环。
然而,此处自环的权重(或相关联的时间常数)由\firstgls{forget_gate}~$f_i^{(t)}$控制(时刻$t$和细胞$i$),由~\ENNAME{sigmoid}~单元将权重设置为0和1之间的值:

\begin{equation}
f_i^{(t)} = \sigma \Big( b_i^f + \sum_j U_{i,j}^f x_j^{(t)} + \sum_j W_{i,j}^f h_j^{(t-1)} \Big),
\end{equation}

其中$x^{(t)}$是当前输入向量,$h^{t}$是当前隐藏层向量,$h^{t}$包含所有~\glssymbol{LSTM}~细胞的输出。 
$b^f, U^f, W^f$分别是\gls{bias_aff}、输入权重和\gls{forget_gate}的循环权重。
因此~\glssymbol{LSTM}~细胞内部状态以如下方式更新,其中有一个条件的自环权重$f_i^{(t)}$:

\begin{align}
s_i^{(t)} = f_i^{(t)}  s_i^{(t-1)} +  g_i^{(t)}
\sigma \Big( b_i + \sum_j U_{i,j} x_j^{(t)} + \sum_j W_{i,j} h_j^{(t-1)} \Big),
\end{align}

其中$b, U, W$分别是~\glssymbol{LSTM}~细胞中的\gls{bias_aff}、输入权重和\gls{forget_gate}的循环权重。
\textbf{外部输入门}(external input gate)单元$g_i^{(t)}$以类似\gls{forget_gate}(使用\ENNAME{sigmoid}获得一个0和1之间的值)的方式更新,但有自身的参数:

\begin{align}
g_i^{(t)} = \sigma \Big( b_i^g + \sum_j U_{i,j}^g x_j^{(t)} + \sum_j W_{i,j}^g h_j^{(t-1)} \Big).
\end{align}

\glssymbol{LSTM}~细胞的输出$h_i^{(t)}$也可以由\textbf{输出门}(output gate)~$q_i^{(t)}$关闭(使用\ENNAME{sigmoid}单元作为\gls{gated}):

\begin{align}
h_i^{(t)} &= \text{tanh}\big( s_i^{(t)} \big) q_i^{(t)}, \\
q_i^{(t)} &= \sigma \Big( b_i^o + \sum_j U_{i,j}^o x_j^{(t)} + \sum_j W_{i,j}^o h_j^{(t-1)} \Big),
\end{align}

其中$b^o, U^o, W^o$分别是\gls{bias_aff}、输入权重和\gls{forget_gate}的循环权重。
在这些变体中,可以选择使用细胞状态$s_i^{(t)}$作为额外的输入(及其权重),输入到第$i$个单元的三个门,如\figref{fig:chapter2_lstm_cell}所示。
这将需要三个额外的参数。

\glssymbol{LSTM}~网络比简单的循环架构更易于学习\gls{long_term_dependency},先是用于测试\gls{long_term_dependency}学习能力的人工数据集\citep{Bengio-trnn94,Hochreiter+Schmidhuber-1997,chapter-gradient-flow-2001},然后是在具有挑战性的序列处理任务上获得最先进的表现\citep{Graves-book2012,Graves-arxiv2013,Sutskever-et-al-NIPS2014}。

\section{神经网络优化算法}
用于深度模型训练的优化算法与传统的优化算法在几个方面有所不同。 机器学习通常是间接作用的。
在大多数机器学习问题中,我们关注某些性能度量 P,其定义于测试集上并且可能是不可解的。
因此,我们只是间接地优化 P。我们希望通过降低代价函数$J(θ)$来提高 P。
这一点与纯优化不同,纯优化最小化目标 J 本身。训练深度模型的优化算法通常也会包括一些针对机器学习目标函数的特定结构进行的特化。

\subsection{\glsentrytext{SGD}优化算法}
\label{sec:stochastic_gradient_descent_chap8}
\gls{SGD}及其变种很可能是一般\gls{ML}中应用最多的优化算法,特别是在\gls{DL}中。
按照\gls{DGD}抽取$m$个\gls{minibatch}(独立同分布的)样本,通过计算它们梯度均值,
我们可以得到梯度的\gls{unbiased}估计。
展示了如何沿着这个梯度的估计下降。

\begin{algorithm}[ht]
	\caption{\gls{SGD}(\glssymbol{SGD})在第$k$个训练迭代的更新}
	\label{alg:sgd}
	\begin{algorithmic}
		\REQUIRE \gls{learning_rate} $\epsilon_k$
		\REQUIRE 初始参数$\theta$
		\WHILE{停止\gls{criterion}未满足}
		\STATE 从\gls{training_set}中采包含$m$个样本$\{ x^{(1)},\dots, x^{(m)}\}$ 的\gls{minibatch},其中$x^{(i)}$对应目标为$y^{(i)}$
		\STATE 计算梯度估计: $\hat{g} \leftarrow + 
		\frac{1}{m} \nabla_{\theta} \sum_i L(f(x^{(i)};\theta),y^{(i)})$
		\STATE 应用更新:$\theta \leftarrow theta - \epsilon \hat{g}$
		\ENDWHILE
	\end{algorithmic}
\end{algorithm}

\glssymbol{SGD}\,算法中的一个关键参数是\gls{learning_rate}。
之前,我们介绍的\,\glssymbol{SGD}\,使用固定的\gls{learning_rate}。
在实践中,有必要随着时间的推移逐渐降低\gls{learning_rate},
因此我们将第$k$步迭代的\gls{learning_rate}记作$\epsilon_k$。

这是因为\,\glssymbol{SGD}\,中梯度估计引入的噪声源($m$个训练样本的随机采样)并不会在\gls{minimum}处消失。
相比之下,当我们使用\gls{batch}\gls{GD}到达\gls{minimum}时,整个\gls{cost_function}的真实梯度会变得很小,
之后为$\mathbf{0}$,因此\gls{batch}\gls{GD}可以使用固定的\gls{learning_rate}。
保证\,\glssymbol{SGD}\,收敛的一个充分条件是
\begin{equation}
\label{eq:8.12}
\sum_{k=1}^\infty \epsilon_k = \infty,
\end{equation}
且
\begin{equation}
\label{eq:8.13}
\sum_{k=1}^\infty \epsilon_k^2 < \infty.
\end{equation}

实践中,一般会线性衰减\gls{learning_rate}直到第$\tau$次迭代:
\begin{equation}
\label{eq:8.14}
\epsilon_k = (1-\alpha) \epsilon_0 + \alpha \epsilon_\tau
\end{equation}
其中$\alpha = \frac{k}{\tau}$。
在$\tau$步迭代之后,一般使$\epsilon$保持常数。

\subsection{RMSProp优化算法}
\label{sec:rmsprop}
\textbf{RMSProp}算法\citep{Hinton-ipam2012}修改AdaGrad以在\gls{nonconvex}设定下效果更好,
改变梯度积累为指数加权的移动平均。
\gls{adagrad}\,旨在应用于凸问题时快速收敛。
当应用于\gls{nonconvex}函数训练\gls{NN}时,学习轨迹可能穿过了很多不同的结构,最终到达一个局部是凸碗的区域。
AdaGrad根据平方梯度的整个历史收缩\gls{learning_rate},可能使得\gls{learning_rate}在达到这样的凸结构前就变得太小了。
RMSProp使用指数衰减平均以丢弃遥远过去的历史,使其能够在找到凸碗状结构后快速收敛,
它就像一个初始化于该碗状结构的AdaGrad算法实例。

RMSProp的标准形式如\algref{alg:rms_prop}所示,结合Nesterov动量的形式如\algref{alg:rms_nesterov}所示。
相比于AdaGrad,使用移动平均引入了一个新的\gls{hyperparameter}$\rho$,用来控制移动平均的长度范围。

经验上,RMSProp已被证明是一种有效且实用的\gls{DNN}优化算法。
目前它是\gls{DL}从业者经常采用的优化方法之一。


\begin{algorithm}[ht]
	\caption{RMSProp算法}
	\label{alg:rms_prop}
	\begin{algorithmic}
		\REQUIRE 全局\gls{learning_rate} $\epsilon$,衰减速率$\rho$
		\REQUIRE  初始参数$\theta$
		\REQUIRE 小常数$\delta$,通常设为$10^{-6}$(用于被小数除时的数值稳定)
		\STATE 初始化累积变量 $r = 0$
		\WHILE{没有达到停止\gls{criterion}}
		\STATE 从\gls{training_set}中采包含$m$个样本$\{ x^{(1)},\dots, x^{(m)}\}$ 的\gls{minibatch},对应目标为$y^{(i)}$。
		\STATE 计算梯度:$g \leftarrow  
		\frac{1}{m} \nabla_{\theta} \sum_i L(f(x^{(i)};\theta),y^{(i)})$ 
		\STATE 累积平方梯度:$r \leftarrow \rho
		r + (1-\rho) g \odot g$
		\STATE 计算参数更新:$\Delta \theta =
		-\frac{\epsilon}{\sqrt{\delta + r}} \odot g$  \ \  ($\frac{1}{\sqrt{\delta + r}}$ 逐元素应用)
		\STATE 应用更新:$\theta \leftarrow \theta + \Delta \theta$
		\ENDWHILE
	\end{algorithmic}
\end{algorithm}

\subsection{Adam}
\label{sec:adam}
\textbf{Adam}是另一种\gls{learning_rate}自适应的优化算法,如\algref{alg:adam}所示。
``Adam''这个名字派生自短语``adaptive moments''。
早期算法背景下,它也许最好被看作结合RMSProp和具有一些重要区别的\gls{momentum}的变种。
首先,在Adam中,\gls{momentum}直接并入了梯度一阶矩(指数加权)的估计。
将\gls{momentum}加入RMSProp最直观的方法是将\gls{momentum}应用于缩放后的梯度。
结合缩放的\gls{momentum}使用没有明确的理论动机。
其次,Adam包括偏置修正,修正从原点初始化的一阶矩(\gls{momentum}项)和(非中心的)二阶矩的估计(\algref{alg:adam})。
RMSProp也采用了(非中心的)二阶矩估计,然而缺失了修正因子。
因此,不像Adam,RMSProp二阶矩估计可能在训练初期有很高的偏置。
Adam通常被认为对\gls{hyperparameter}的选择相当鲁棒,尽管\gls{learning_rate}有时需要从建议的默认修改。

\begin{algorithm}[ht]
	\caption{Adam算法}
	\label{alg:adam}
	\begin{algorithmic}
		\REQUIRE 步长 $\epsilon$ (建议默认为: $0.001$)
		\REQUIRE 矩估计的指数衰减速率, $\rho_1$ 和 $\rho_2$ 在区间 $[0, 1)$内。
		(建议默认为:分别为$0.9$ 和 $0.999$)
		\REQUIRE 用于数值稳定的小常数 $\delta$  (建议默认为: $10^{-8}$)
		\REQUIRE 初始参数 $\theta$
		\STATE 初始化一阶和二阶矩变量 $s = 0 $, $r = 0$
		\STATE 初始化\gls{time_step} $t=0$ 
		\WHILE{没有达到停止\gls{criterion}}
		\STATE 从\gls{training_set}中采包含$m$个样本$\{ x^{(1)},\dots, x^{(m)}\}$ 的\gls{minibatch},对应目标为$y^{(i)}$。
		\STATE 计算梯度:$g \leftarrow \frac{1}{m} \nabla_{\theta} \sum_i L(f(x^{(i)};\theta),y^{(i)})$ 
		\STATE $t \leftarrow t + 1$
		\STATE 更新有偏一阶矩估计: $s \leftarrow \rho_1 s + (1-\rho_1) g$
		\STATE 更新有偏二阶矩估计:$r \leftarrow \rho_2 r + (1-\rho_2)  g \odot g$
		\STATE 修正一阶矩的\gls{bias_sta}:$\hat{s} \leftarrow \frac{s}{1-\rho_1^t}$
		\STATE 修正二阶矩的\gls{bias_sta}:$\hat{r} \leftarrow \frac{r}{1-\rho_2^t}$
		\STATE 计算更新:$\Delta \theta = - \epsilon \frac{\hat{s}}{\sqrt{\hat{r}} + \delta}$ \ \  (逐元素应用操作)
		\STATE 应用更新:$\theta \leftarrow \theta + \Delta \theta$
		\ENDWHILE
	\end{algorithmic}
\end{algorithm}

\subsection{选择正确的优化算法}
\label{sec:choosing_the_right_optimization_algorithms}
在本节中,我们讨论了一系列算法,通过自适应每个模型参数的\gls{learning_rate}以解决优化\gls{deep_model}中的难题。
此时,一个自然的问题是:该选择哪种算法呢?

遗憾的是,目前在这一点上没有达成共识。
\cite{Schaul2014_unittests}展示了许多优化算法在大量学习任务上极具价值的比较。
虽然结果表明,具有自适应\gls{learning_rate}(以RMSProp和AdaDelta为代表)的算法族表现得相当鲁棒,不分伯仲,但没有哪个算法能脱颖而出。

目前,最流行并且使用很高的优化算法包括SGD、具\gls{momentum}的SGD、RMSProp、具\gls{momentum}的RMSProp、AdaDelta和Adam。
此时,选择哪一个算法似乎主要取决于使用者对算法的熟悉程度(以便调节\gls{hyperparameter})。

\section{本章小结}