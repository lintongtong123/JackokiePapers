\documentclass{article}
\usepackage{ctex,hyperref}
\usepackage{geometry}
\geometry{paperwidth=21cm,paperheight=26cm,%
left=1cm,right=1cm,top=1cm,bottom=1.5cm}
\usepackage[backend=biber,style=gb7714-2015]{biblatex}
\addbibresource{example.bib}
\renewcommand{\bibfont}{\zihao{6}}
\usepackage{titlesec}
%\titleformat{command}[shape]{format}{label}{sep}{before}[after]
\titleformat{\section}{\centering\bfseries}{第\thesection 节}{1em}{}[]
\titlespacing*{\section}{0pt}{0.0\baselineskip}{0.0\baselineskip}[0pt]
\titleformat{\subsection}{\flushleft\bfseries}{\S\,\thesubsection}{1em}{}[]
\titlespacing*{\subsection}{0pt}{0.0\baselineskip}{0.0\baselineskip}[0pt]
\begin{document}
\small 	参考文献测试\cite{Gradshteyn2000--}。
	\begin{refsegment}
		\section{refSegment A}
		分章节参考文献测试\cite{Chiani2003-840-845}
		\printbibliography[segment=1,heading=subbibliography,title=文献A]
	\end{refsegment}
	
	\begin{refsegment}
		\section{refSegment B}
		参考文献测试\cite{张敏莉2007-500-503}
	\end{refsegment}
	%\printbibliography放在refsegment环境外也是可以的
	\printbibliography[segment=2,heading=subbibliography,title=文献B]
	
	\begin{refsection}
		\section{refsection C}
		参考文献测试\cite{Zhang2007-500-503}
	\end{refsection}

	\begin{refsection}
		\section{refsection D}
		分章节参考文献测试\cite{Andersen1995-42-49}
    	\begin{refsegment}
			\subsection{refsegment D-1}
			分章节参考文献测试\cite{Simon2004--}。
		\end{refsegment}
		\begin{refsegment}
			\subsection{refsegment D-2}
			分章节参考文献测试\cite{Lin2004--}。
		\end{refsegment}
	\end{refsection}
    \printbibliography[section=2,segment=0,heading=subbibliography,title=文献D0]
    \printbibliography[section=2,segment=1,heading=subbibliography,title=文献D1]
    \printbibliography[section=2,segment=2,heading=subbibliography,title=文献D2]
	\printbibliography[section=1,heading=subbibliography,title=文献C]
    \printbibliography[section=2,heading=subbibliography,title=文献D]
	%遍历非refsection内的参考文献
	\printbibliography[heading=bibliography,title=文献全局]
	\appendix
	\section{Section E}
	参考文献测试\cite{Parsons2000--}。
\end{document} 